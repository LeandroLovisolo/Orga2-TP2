El filtro umbralizar consiste en evaluar cada pixel de la imagen y asignarle un nuevo valor según tres criterio.

\begin{itemize}
\item Si el pixel supera el máximo pasado por parámetro a la función, se le coloca un 255.
\item Si el pixel es menor que el mínimo, también pasado por parámetro, se le coloca un 0.
\item Si el pixel está entre ($min \leq pixel \leq max$), se le asigna $\lfloor pixel/Q \rfloor * Q$, donde $Q$ es un parámetro.
\end{itemize}


\subsubsection{Descripción del ciclo}

El filtro se divide en cinco etapas:

\begin{enumerate}
\item \textbf{Pre-ciclo:} Se crean ciertos registros que serán de utilidad en el ciclo.
\item \textbf{Inicio del ciclo:} Puesta en 0 del registro acumulador y obtención de máscara de mínimos.
\item \textbf{Obtención de la máscara para píxeles mayores al máximo y aplicación:} Se consigue la máscara para los píxeles que superen al mayor y se aplica al acumulador.
\item \textbf{Creación de la máscara para los píxeles ($min \leq pixel \leq max$):} Se arma la máscara a partir de una nueva comparación y máscaras anteriores.
\item \textbf{Aplicación de la máscara para ($min \leq pixel \leq max$) y fin del ciclo:} Se realizan los cálculos pertinentes a los píxeles que entran en esta categoría y lo aplica a la máscara.
\end{enumerate}


\subsubsection{Pre-ciclo}

Para optimizar el procesamiento dentro del ciclo, se calcula y se guarda en registros XMM al inicio del programa:
\begin{itemize}
  \item \xmm{12} $\leftarrow$ Contiene 16 bytes packed con el mínimo
  \item \xmm{11} $\leftarrow$ Contiene el mínimo packed en words
  \item \xmm{5}  $\leftarrow$ Contiene el máximo packed en words
  \item \xmm{6}  $\leftarrow$ Contiene la representación flotante de Q packed
\end{itemize}

También se calcula y se guarda en el registro \rcx, la cantidad de píxeles que tiene la imagen, información que se utilizará para conocer cuántos píxeles faltan por procesar en cada iteración.

\subsubsection{Inicio del ciclo}
Al comienzo del ciclo se ponen en 0 los bytes del registro \xmm{8} que servirá como acumulador de los nuevos valores que tendrán los píxeles.

Se leen 16 bytes contiguos desde la imagen fuente en el registro \xmm{1} utilizando la instrucción \texttt{MOVDQU}, se busca cuáles son iguales al mínimo (aprovechando la instrucción \asm{PCMPEQB} que realiza la comparación en los 16 bytes simultáneamente) y se guarda la máscara obtenida que se usará mas tarde.

\subsubsection{Obtención de la máscara para píxeles mayores al máximo y aplicación}

Queriendo explotar al máximo las instrucciones SIMD, tratando de procesar en simultaneo 16 bytes, encontramos que esto no era posible debido a que el set de instrucciones no contempla comparaciones de greater, lower o derivadas para bytes sin signo (necesario ya que los píxeles en grayscale van del 0 al 255), por lo que fue necesario extender los bytes mediante un desempaquetado, convirtiéndolos en words, para luego hacer las comparaciones correspondientes.

Se desempaquetan los 16 bytes en parte baja (primeros 8 bytes) y en parte alta convirtiéndolos a words por medio de la instrucción \texttt{PUNPCKLBW}.

\begin{figure}[H]
  \centering
  \begin{tikzpicture}[scale=0.75]
    \registroDieciseis{\xmm{1}}{0}{6}
                      {A15}{A14}{A13}{A12}{A11}{A10}{A9}{A8}{A7}
                      {A6}{A5}{A4}{A3}{A2}{A1}{A0}

    \draw [decoration={brace,amplitude=0.25cm},decorate, thick] ( 7.75, 5.75) -- (0.25, 5.75);
    \draw [decoration={brace,amplitude=0.25cm},decorate, thick] (15.75, 5.75) -- (8.25, 5.75);

    \draw [->, thick] (12, 5.5)  --  +(0, -1.25);
    \draw [->, thick] (4, 5.5)   -- ++(0, -0.75) -- ++(-6, 0) --
                    ++(0, -2.75) -- ++(6, 0)     --  +(0, -0.75);

    \registroDieciseis{\xmm{1}}{0}{3}
                      {0}{A7}{0}{A6}{0}{A5}{0}{A4}{0}{A3}{0}{A2}{0}{A1}{0}{A0}

    \registroDieciseis{\xmm{2}}{0}{0}
                      {0}{A15}{0}{A14}{0}{A13}{0}{A12}{0}{A11}{0}{A10}{0}{A9}{0}{A8}
  \end{tikzpicture}
  \caption{Desempaquetamiento con la instrucción \asm{PUNPCKLBW}.}
\end{figure}

Se buscan los números que superen al máximo comparando la parte alta y baja con el registro \xmm{5} preparado al inicio del programa, el cuál contiene el máximo en words empaquetadas. Utilizando \texttt{PCMPGTW} tanto en la parte baja como la alta, obtenemos una máscara que contiene en \texttt{0xFFFF} las words mayores al máximo y en \texttt{0x0000} las demás. Empaquetamos, utilizando la instrucción \texttt{PACKSSWB}, la máscara relacionada a la parte baja y a la alta, dejando en 255 sólo los bytes que sean mayores al máximo para luego sumarlos en el acumulador utilizando \texttt{PADDUSB}.

\begin{figure}[H]
  \centering
  \begin{tikzpicture}[scale=0.75]
    \draw (-1, 15) node[anchor=west] {Estado inicial de los registros:};

    \registroDieciseis{\xmm{5}}{0}{13}
                      {0}{Max}{0}{Max}{0}{Max}{0}{Max}{0}{Max}{0}{Max}{0}{Max}{0}{Max}

    \registroDieciseis{\xmm{4}}{0}{11}
                      {0}{A15}{0}{A14}{0}{A13}{0}{A12}{0}{A11}{0}{A10}{0}{A9}{0}{A8}

    \registroDieciseis{\xmm{3}}{0}{9}
                      {0}{A7}{0}{A6}{0}{A5}{0}{A4}{0}{A3}{0}{A2}{0}{A1}{0}{A0}

    \draw (-1, 8) node[anchor=west]
      {Luego de ejecutar \texttt{PCMPGTW} \xmm{3}, \xmm{5} y \texttt{PCMPGTW} \xmm{4}, \xmm{5}:};

    \registroDieciseis{\xmm{4}}{-1}{6}
                      {0}{0}{0}{0}{0}{0}{0}{0}{0}{0}{0}{0}{FF}{FF}{FF}{FF}

    \registroDieciseis{\xmm{3}}{1}{4}
                      {0}{0}{0}{0}{FF}{FF}{FF}{FF}{0}{0}{FF}{FF}{0}{0}{FF}{FF}

    \draw [->, thick] (12, 3.75)   --  +(0, -1.75);

    \draw [->, thick] (-0.5, 5.75) -- ++(0, -3.25) -- ++ (4.5, 0) -- + (0, -0.5);

    \draw [decoration={brace,amplitude=0.25cm},decorate, thick] (0.25, 1.25) -- ( 7.75, 1.25);
    \draw [decoration={brace,amplitude=0.25cm},decorate, thick] (8.25, 1.25) -- (15.75, 1.25);

    \draw (-1, 3) node[anchor=west, fill=white]
      {Luego de ejecutar \texttt{PACKSSWB} \xmm{3}, \xmm{4}:};

    \registroDieciseis{\xmm{3}}{0}{0}
                      {0}{0}{0}{0}{0}{0}{FF}{FF}{0}{0}{FF}{FF}{0}{FF}{0}{FF}

  \end{tikzpicture}
  \caption{Búsqueda de máximos.}
\end{figure}



\subsubsection{Creación de la máscara para los píxeles ($min \leq pixel \leq max$)}
Al utilizar un acumulador incialmente en 0 se pudo ahorrar el paso de comparar los píxeles menores al mínimo ya que los píxeles que cumplieran esta propiedad tendrían 0 como valor.

Para conseguir la máscara, se aprovechó las máscaras previamente obtenidas (mayores al máximo e iguales al mínimo), por lo que sólo fue necesario buscar los números mayores al mínimo (haciendo un procedimiento similar al de la búsqueda del máximo), agregarle los números iguales al mínimo (mediante un \texttt{POR}) con la máscara creada al principio del ciclo y luego sacarle los mayores al máximo (mediante un \texttt{PXOR}) dejándome sólo los píxeles que cumplen esta condición.

\subsubsection{Aplicación de la máscara para ($min \leq pixel \leq max$) y fin del ciclo}
Al ser necesaria una división y un truncamiento para el caso, se utilizaron single precision floats.

Fue necesario desempaquetar aún más los píxeles, utilizando \texttt{PUNPCKLWD}, para poder obtener los valores de estos en double words, y poder convertirlos a single precision floats mediante la instrucción \texttt{CVTDQ2PS}. Esta es precisión suficiente para los cálculos y además brinda la posibilidad realizar más operaciones sobre los píxeles simultáneamente que con doubles.

Se realiza el desempaquetado de la parte baja anteriormente desempaquetada (\xmm{1}) y se los convierte a single presicion floats. Luego se los divide, utilizando \texttt{DIVPS}, por el registro \xmm{6} el cuál contiene el valor Q empaquetado en single presicion floats calculado al inicio del programa. Se lo trunca utilizando \texttt{CVTTPS2DQ}, para realizar la función floor ($\lfloor \rfloor$) convirtiéndose en entero, se lo convierte otra vez a float y se lo termina multiplicando, utilizando \texttt{MULPS}, por \xmm{6} otra vez para finalmente ser convertido a entero, obteniendo el valor correspondiente para cada pixel. Se vuelve a empaquetar, utilizando la instrucción \texttt{PACKUSDW}, obteniéndose los valores nuevos de los píxeles otra vez en words y se repite la operación con la parte alta del desempaquetado inicial (\xmm{2}).

Al finalizar, se empaquetan los dos registros resultantes, mediante \texttt{PACKUSWB}, para luego poder aplicarle la máscara anteriormente creada para el caso utilizando un \texttt{PAND} y sumar simultáneamente con \texttt{PADDUSB} el valor obtenido al acumulador, dejando el valor correspondiente en los píxeles que cumplían con el caso.

Por último, se copian los nuevos valores de los 16 bytes en la imagen destino utilizando \texttt{MOVDQU} y se incrementan los punteros de la imagen fuente y destino para la próxima iteración. También se resta nuestro contador de píxeles por procesar, se lo compara para ver si llegó al final, caso en el que termina la ejecución, o si quedan más e igual, caso en el que vuelve a ciclar, o si restan menos de 16 bytes por agarrar, dónde se vuelve para atrás los punteros de las imágenes para que queden 16 píxeles exactamente y se pueda realizar la última iteración.

\subsubsection{Comparación con la implementación C}
El ciclo en C hace esencialmente lo mismo que la implementación en assembler, pero esta lo hace de manera simultanea utilizando SIMD. 

Estos son los detalles de las operaciones realizadas en C para 16 bytes:
\begin{itemize}
\item 16 accesos a memoria para la lectura de cada byte y 16 para la escritura
\item Se realizan a lo sumo 32 comparaciones, (2 por cada pixel), en dónde se necesita calcular la posición del pixel cada vez que se lo quiere leer y se realiza un acceso a memoria para esto.
\item En caso de haber caído dentro del caso ($min \leq pixel \leq max$), se realizan cada una de las operaciones de cálculo de manera secuencial.
\end{itemize}
Detalles de las operaciones realizadas por la implementación de assembler en 16 bytes:
\begin{itemize}
\item Única lectura y escritura en memoria de 16 bytes
\item Se realizan 2 comparaciones para el caso de los píxeles mayores al máximo y para calcular los píxeles mayores al mínimo por cada 16 bytes, ya que se comparan de a 8 bytes simultáneamente utilizando SIMD. También se realiza 1 comparación cada estos 16 bytes para obtener los iguales al mínimo.
\item Si bien siempre se realiza el cálculo en punto flotante relacionado a los valores de los píxeles del caso ($min \leq pixel \leq max$), estos se realizan con 4 instrucciones SIMD, pudiendo calcular el valor de 16 bytes por ciclo.
\end{itemize}

\subsubsection{Rendimiento}
Observamos las siguientes cantidades de ciclos y ticks de reloj al realizar 100 iteraciones de ambas implementaciones con una imagen cuadrada de lado 512 y parámetros $min = 64$, $max = 128$ y $Q = 16$.
\begin{center}
    \begin{tabular}{|l|l|l|l|}
        \hline
        Medición & Implementación C & Implementación assembler & Relación \\
        \hline
        Ticks    & 785867600      & 70172072               & $8.92\%$ \\
        Ciclos   & 7858676        & 701721                & $8.92\%$ \\
        \hline
    \end{tabular}
\end{center}

Para hacer un análisis fino, utilizamos la herramienta objdump (\texttt{objdump -d -M intel -S umbralizar\_c.o}) para obtener el resultado de la compilación mediante gcc.

Destacamos de este análisis el uso contínuo de variables locales (almacenadas en el stack) y el cálculo de la posición en memoria y el acceso a esta cada vez que se quería realizar una comparación, algo evitable usando registros en assembler y que seguro impactaron en el rendimiento.

\begin{verbatim}
if(src_matrix[y][x] < min) {
;Instrucciones para settear la posición del byte
mov    eax,DWORD PTR [rbp-0x4c]
movsxd rdx,eax
movsxd rax,ebx
imul   rdx,rax
mov    rax,QWORD PTR [rbp-0x38]
add    rdx,rax
mov    eax,DWORD PTR [rbp-0x48]
cdqe   
movzx  eax,BYTE PTR [rdx+rax*1]
;Termina el setteado de la posición del byte
cmp    al,BYTE PTR [rbp-0x70] ;Realiza la comparación
jae    c3 <umbralizar_c+0xc3>
\end{verbatim}

Cabe destacar que el lenguaje assembler, además del uso de las instrucciones SIMD, que nos permiten realizar cálculos simultáneamente otorgándonos una gran ventaja contra el código en C, también nos permite explotar los recursos al máximo, pudiendo obviar cálculos de más guardándolos para su posterior uso y utilizar registros que implican menos cantidad de accesos a memoria.

Estas ventajas, sumadas a los pocos accesos de memoria al poder escribir y leer de a 16 bytes son la causa del bajo tiempo de ejecución del filtro en assembler.