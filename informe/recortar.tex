La salida del filtro recortar consiste en, dada una longitud $n$, recortar las esquinas de la imagen
original en forma de cuadrados de lado $n$ y unirlos para obtener la imagen destino, ubicando cada
cuadrado en su posición opuesta respecto de la imagen original.

\subsubsection{Descripción del ciclo}

Se ilustra a continuación el ciclo de la implementación C:

\begin{pseudocodigo}
    \FOR{$y = 1$ to $n$}
        \FOR{$x = 1$ to $n$}
            \STATE $dst[alto - n + y][ancho - n + x]$ $\leftarrow$ $src[y][x]$ \COMMENT{copio esquina superior izquierda}
            \STATE $dst[alto - n + y][x]$ $\leftarrow$ $src[y][ancho - n + x]$ \COMMENT{copio esquina superior derecha}
            \STATE $dst[y][ancho - n + x]$ $\leftarrow$ $src[alto - n + y][x]$ \COMMENT{copio esquina inferior izquierda}
            \STATE $dst[y][x]$ $\leftarrow$ $src[alto - n + y][ancho - n + x]$ \COMMENT{copio esquina inferior derecha}
        \ENDFOR
    \ENDFOR
\end{pseudocodigo}

La implementación assembler difiere de la anterior en cuanto a que se copia la imagen de a 16 bytes por vez,
utilizando la instrucción \texttt{MOVDQU} para cargar 16 bytes de memoria contiguos en un registro XMM 
y luego volcar el contenido del mismo en la posición de memoria correspondiente.

Además, se realiza un ajuste sobre la variable $x$ para evitar leer y escribir posiciones de memoria fuera
de la imagen y/o del cuadrado correspondiente: cuando la diferencia $|n - x|$ es menor a 16, se le resta
esta diferencia a $x$, de manera que el siguiente ciclo opere exactamente con los 16 últimos bytes de la fila actual.

El pseudocódigo a continuación describe el ciclo completo de la implementación assembler:

\begin{pseudocodigo}
    \FOR{$y = 1$ to $n$}
        \FOR{$x = 1$ to $n$}
            \IF{$|n - x| < 16$}
                \STATE $x$ $\leftarrow$ $x - |n - x|$
            \ENDIF
            \STATE \texttt{MOVDQU XMM0}, $src[y][x]$ \COMMENT{copio esquina superior izquierda}
            \STATE \texttt{MOVDQU} $dst[alto - n + y][ancho - n + x]$, \texttt{XMM0} \COMMENT{pego esquina superior izquierda}            
            \STATE idem demás esquinas
        \ENDFOR
    \ENDFOR
\end{pseudocodigo}

\subsubsection{Rendimiento}

Se observan las siguientes cantidades de ciclos y ticks de reloj al realizar 1000 iteraciones de ambas implementaciones con una imagen cuadrada de lado 512 y tamaño de esquina 100.

\begin{center}
    \begin{tabular}{|l|l|l|l|}
        \hline
        Medición & Implementación C & Implementación assembler & Relación \\
        \hline
        Ticks    & 385953030        & 23853906                 & $6.43\%$ \\
        Ciclos   & 378948.344       & 24397.717                & $6.43\%$ \\
        \hline
    \end{tabular}
\end{center}

Dado que la implementación assembler accede a la memoria de a 16 bytes por vez, la cantidad de
accesos totales será aproximadamente $\frac{1}{16} = 0.0625 = 6.25\%$ la cantidad de
accesos a memoria de la implementación C (no es exactamente esa cantidad pues los ajustes de
fin de línea introducen un pequeño adicional de lecturas, dependiendo del ancho de la imagen.)
Notemos que este número se condice con las mediciones de rendimiento realizadas.