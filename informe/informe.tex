\documentclass[a4paper,10pt,twoside]{article}

\usepackage[top=1in, bottom=1in, left=1in, right=1in]{geometry}
\usepackage[utf8]{inputenc}
\usepackage[spanish,es-ucroman,es-noquoting]{babel}
\usepackage{setspace}
\usepackage{fancyhdr}
\usepackage{lastpage}
\usepackage{amsmath}
\usepackage{amsfonts}
\usepackage{verbatim}
\usepackage{graphicx}
\usepackage{float}
\usepackage{algorithmic}
\usepackage{tikz}
\usetikzlibrary{calc}
\usetikzlibrary{decorations.pathreplacing}


% Evita que el documento se estire verticalmente para ocupar
% el espacio vacío en cada página.
\raggedbottom


%%%%%%%%%% Configuración de Fancyhdr - Inicio %%%%%%%%%%
\pagestyle{fancy}
\thispagestyle{fancy}
\lhead{Trabajo Práctico 2, Organización del Computador II}
\rhead{Belloli, Lovisolo, Petaccio}
\renewcommand{\footrulewidth}{0.4pt}
\cfoot{\thepage /\pageref{LastPage}}

\fancypagestyle{caratula} {
   \fancyhf{}
   \cfoot{\thepage /\pageref{LastPage}}
   \renewcommand{\headrulewidth}{0pt}
   \renewcommand{\footrulewidth}{0pt}
}
%%%%%%%%%% Configuración de Fancyhdr - Fin %%%%%%%%%%


%%%%%%%%%% Configuración de Algorithmic - Inicio %%%%%%%%%%
% Entorno propio para customizar la presentación del pseudocódigo
\newenvironment{pseudocodigo}
    {\vspace{0.5em} \begin{algorithmic}}
    {\end{algorithmic} \vspace{0.5em}}

% Alinear comentarios a la derecha
\renewcommand{\algorithmiccomment}[1]{\hfill \{#1\}}
%%%%%%%%%% Configuración de Algorithmic - Fin %%%%%%%%%%


%%%%%%%%%% Macros de tikz - Inicio %%%%%%%%%%
% Uso: \registroCuatro{etiqueta}{x}{y}{a4}{a3}{a2}{a1}
\newcommand{\registroCuatro}[7]{
    \ifthenelse{\equal{#1}{}}{}{
        \draw (#2, {#3 + 0.5}) node[anchor=east]{#1};
    }

    \draw   (#2, #3) rectangle +(4, 1) +(2, 0.5) node{#4}
          ++(4, 0)   rectangle +(4, 1) +(2, 0.5) node{#5}
          ++(4, 0)   rectangle +(4, 1) +(2, 0.5) node{#6}
          ++(4, 0)   rectangle +(4, 1) +(2, 0.5) node{#7};          
}

% Uso: \registroOcho{etiqueta}{x}{y}{a8}{a7}{a6}...{a1}
\newcommand{\registroOcho}[9]{
    \def\etiqueta{#1}
    \def\x{#2}
    \def\y{#3}
    \def\aviii{#4}
    \def\avii{#5}
    \def\avi{#6}
    \def\av{#7}
    \def\aiv{#8}
    \def\aiii{#9}
    \registroOchoX    
}
\newcommand{\registroOchoX}[2]{ % Auxiliar - no usar directamente
    \def\aii{#1}
    \def\ai{#2}
    \ifthenelse{\equal{\etiqueta}{}}{}{
        \draw (\x, {\y + 0.5}) node[anchor=east]{\etiqueta};
    }
    \filldraw[fill=white]
        (\x, \y) rectangle +(2, 1) +(1, 0.5) node{\aviii}
        ++(2, 0) rectangle +(2, 1) +(1, 0.5) node{\avii}
        ++(2, 0) rectangle +(2, 1) +(1, 0.5) node{\avi}
        ++(2, 0) rectangle +(2, 1) +(1, 0.5) node{\av}
        ++(2, 0) rectangle +(2, 1) +(1, 0.5) node{\aiv}
        ++(2, 0) rectangle +(2, 1) +(1, 0.5) node{\aiii}
        ++(2, 0) rectangle +(2, 1) +(1, 0.5) node{\aii}
        ++(2, 0) rectangle +(2, 1) +(1, 0.5) node{\ai};
}


% Uso: \registroDieciseis{etiqueta}{x}{y}{a16}{a15}{a14}...{a1}
\newcommand{\registroDieciseis}[9]{
    \def\etiqueta{#1}
    \def\x{#2}
    \def\y{#3}
    \def\axvi{#4}
    \def\axv{#5}
    \def\axiv{#6}
    \def\axiii{#7}
    \def\axii{#8}
    \def\axi{#9}
    \registroDieciseisX
}
\newcommand{\registroDieciseisX}[9]{ % Auxiliar - no usar directamente
    \def\ax{#1}
    \def\aix{#2}
    \def\aviii{#3}
    \def\avii{#4}
    \def\avi{#5}
    \def\av{#6}
    \def\aiv{#7}
    \def\aiii{#8}
    \def\aii{#9}
    \registroDieciseisXX
}
\newcommand{\registroDieciseisXX}[1]{ % Auxiliar - no usar directamente
    \def\ai{#1}
    \ifthenelse{\equal{\etiqueta}{}}{}{
        \draw (\x, {\y + 0.5}) node[anchor=east]{\etiqueta};
    }
    \filldraw[fill=white]
        (\x, \y) rectangle +(1, 1) +(0.5, 0.5) node{\axvi}
        ++(1, 0) rectangle +(1, 1) +(0.5, 0.5) node{\axv}
        ++(1, 0) rectangle +(1, 1) +(0.5, 0.5) node{\axiv}
        ++(1, 0) rectangle +(1, 1) +(0.5, 0.5) node{\axiii}
        ++(1, 0) rectangle +(1, 1) +(0.5, 0.5) node{\axii}
        ++(1, 0) rectangle +(1, 1) +(0.5, 0.5) node{\axi}
        ++(1, 0) rectangle +(1, 1) +(0.5, 0.5) node{\ax}
        ++(1, 0) rectangle +(1, 1) +(0.5, 0.5) node{\aix}
        ++(1, 0) rectangle +(1, 1) +(0.5, 0.5) node{\aviii}
        ++(1, 0) rectangle +(1, 1) +(0.5, 0.5) node{\avii}
        ++(1, 0) rectangle +(1, 1) +(0.5, 0.5) node{\avi}
        ++(1, 0) rectangle +(1, 1) +(0.5, 0.5) node{\av}
        ++(1, 0) rectangle +(1, 1) +(0.5, 0.5) node{\aiv}
        ++(1, 0) rectangle +(1, 1) +(0.5, 0.5) node{\aiii}
        ++(1, 0) rectangle +(1, 1) +(0.5, 0.5) node{\aii}
        ++(1, 0) rectangle +(1, 1) +(0.5, 0.5) node{\ai};
}
%%%%%%%%%% Macros de tikz - Fin %%%%%%%%%%


%%%%%%%%%% Macros misceláneos - Inicio %%%%%%%%%%
\newcommand{\xmm}[1]{\texttt{XMM#1}}
\newcommand{\rax}{\texttt{RAX}}
\newcommand{\rbx}{\texttt{RBX}}
\newcommand{\rcx}{\texttt{RCX}}
\newcommand{\rdx}{\texttt{RDX}}
\newcommand{\rbp}{\texttt{RBP}}
\newcommand{\rsp}{\texttt{RSP}}
\newcommand{\reg}[1]{\texttt{R#1}}
\newcommand{\asm}[1]{\texttt{\uppercase{#1}}}
%%%%%%%%%% Macros misceláneos - Fin %%%%%%%%%%


\begin{document}


%%%%%%%%%%%%%%%%%%%%%%%%%%%%%%%%%%%%%%%%%%%%%%%%%%%%%%%%%%%%%%%%%%%%%%%%%%%%%%%
%% Carátula                                                                  %%
%%%%%%%%%%%%%%%%%%%%%%%%%%%%%%%%%%%%%%%%%%%%%%%%%%%%%%%%%%%%%%%%%%%%%%%%%%%%%%%


\thispagestyle{caratula}

\begin{center}

\includegraphics[height=2cm]{DC.png} 
\hfill
\includegraphics[height=2cm]{UBA.jpg} 

\vspace{2cm}

Departamento de Computación,\\
Facultad de Ciencias Exactas y Naturales,\\
Universidad de Buenos Aires

\vspace{4cm}

\begin{Huge}
Trabajo Práctico 2
\end{Huge}

\vspace{0.5cm}

\begin{Large}
Organización del Computador II
\end{Large}

\vspace{1cm}

Primer Cuatrimestre de 2013

\vspace{4cm}

Grupo: \textbf{Panceta y Mozzarella}

\vspace{0.5cm}

\begin{tabular}{|c|c|c|}
\hline
Apellido y Nombre & LU & E-mail\\
\hline
Laouen Louan Mayal Belloli  & 134/11 & lao.facu@gmail.com\\
Leandro Lovisolo      		& 645/11 & leandro@leandro.me\\
Lautaro José Petaccio 		& 443/11 & lausuper@gmail.com\\
\hline
\end{tabular}

\end{center}

\newpage


%%%%%%%%%%%%%%%%%%%%%%%%%%%%%%%%%%%%%%%%%%%%%%%%%%%%%%%%%%%%%%%%%%%%%%%%%%%%%%%
%% Índice                                                                    %%
%%%%%%%%%%%%%%%%%%%%%%%%%%%%%%%%%%%%%%%%%%%%%%%%%%%%%%%%%%%%%%%%%%%%%%%%%%%%%%%


\tableofcontents

\newpage


%%%%%%%%%%%%%%%%%%%%%%%%%%%%%%%%%%%%%%%%%%%%%%%%%%%%%%%%%%%%%%%%%%%%%%%%%%%%%%%
%% Introducción                                                              %%
%%%%%%%%%%%%%%%%%%%%%%%%%%%%%%%%%%%%%%%%%%%%%%%%%%%%%%%%%%%%%%%%%%%%%%%%%%%%%%%


\section{Introducción}

El objetivo de este trabajo es experimentar con el set de instrucciones SIMD de la arquitectura IA-32 de Intel. 

Con este propósito, implementamos una serie de filtros gráficos que operan sobre imágenes RGB y en escala de grises. Para cada filtro escribimos una implementación de referencia en lenguaje C, sin ningún tipo de optimización, y otra implementación en lenguaje assembler haciendo uso de las instrucciones SIMD siempre que fuera posible. Finalmente, evaluamos el rendimiento de ambas implementaciones de cada filtro, y discutimos las mejoras observadas.

En la siguiente sección analizamos cada una de las implementaciones y describimos en detalle las instrucciones SIMD usadas en cada caso.


%%%%%%%%%%%%%%%%%%%%%%%%%%%%%%%%%%%%%%%%%%%%%%%%%%%%%%%%%%%%%%%%%%%%%%%%%%%%%%%
%% Desarrollo                                                                %%
%%%%%%%%%%%%%%%%%%%%%%%%%%%%%%%%%%%%%%%%%%%%%%%%%%%%%%%%%%%%%%%%%%%%%%%%%%%%%%%


\section{Desarrollo}

\subsection{Recortar}

La salida del filtro recortar consiste en, dada una longitud $n$, recortar las esquinas de la imagen
original en forma de cuadrados de lado $n$ y unirlos para obtener la imagen destino, ubicando cada
cuadrado en su posición opuesta respecto de la imagen original.

\subsubsection{Descripción del ciclo}

Se ilustra a continuación el ciclo de la implementación C:

\begin{algorithmic}
    \FOR{$y = 1$ to $n$}
        \FOR{$x = 1$ to $n$}
            \STATE $dst[alto - n + y][ancho - n + x]$ $\leftarrow$ $src[y][x]$ \COMMENT{copio esquina superior izquierda}
            \STATE $dst[alto - n + y][x]$ $\leftarrow$ $src[y][ancho - n + x]$ \COMMENT{copio esquina superior derecha}
            \STATE $dst[y][ancho - n + x]$ $\leftarrow$ $src[alto - n + y][x]$ \COMMENT{copio esquina inferior izquierda}
            \STATE $dst[y][x]$ $\leftarrow$ $src[alto - n + y][ancho - n + x]$ \COMMENT{copio esquina inferior derecha}
        \ENDFOR
    \ENDFOR
\end{algorithmic}

La implementación assembler difiere de la anterior en cuanto a que se copia la imagen de a 16 bytes por vez,
utilizando la instrucción \texttt{MOVDQU} para cargar 16 bytes de memoria contiguos en un registro XMM 
y luego volcar el contenido del mismo en la posición de memoria correspondiente.

Además, se realiza un ajuste sobre la variable $x$ para evitar leer y escribir posiciones de memoria fuera
de la imagen y/o del cuadrado correspondiente: cuando la diferencia $|n - x|$ es menor a 16, se le resta
esta diferencia a $x$, de manera que el siguiente ciclo opere exactamente con los 16 últimos bytes de la fila actual.

El pseudocódigo a continuación describe el ciclo completo de la implementación assembler:

\begin{algorithmic}
    \FOR{$y = 1$ to $n$}
        \FOR{$x = 1$ to $n$}
            \IF{$|n - x| < 16$}
                \STATE $x$ $\leftarrow$ $x - |n - x|$
            \ENDIF
            \STATE \texttt{MOVDQU XMM0}, $src[y][x]$ \COMMENT{copio esquina superior izquierda}
            \STATE \texttt{MOVDQU} $dst[alto - n + y][ancho - n + x]$, \texttt{XMM0} \COMMENT{pego esquina superior izquierda}            
            \STATE idem demás esquinas
        \ENDFOR
    \ENDFOR
\end{algorithmic}

\subsubsection{Rendimiento}

Se observan las siguientes cantidades de ciclos y ticks de reloj al realizar 1000 iteraciones de ambas implementaciones con una imagen cuadrada de lado 512 y tamaño de esquina 100.

\begin{center}
    \begin{tabular}{|l|l|l|l|}
        \hline
        Medición & Implementación C & Implementación assembler & Relación \\
        \hline
        Ticks    & 385953030        & 23853906                 & $6.43\%$ \\
        Ciclos   & 378948.344       & 24397.717                & $6.43\%$ \\
        \hline
    \end{tabular}
\end{center}

Dado que la implementación assembler accede a la memoria de a 16 bytes por vez, la cantidad de
accesos totales será aproximadamente $\frac{1}{16} = 0.0625 = 6.25\%$ la cantidad de
accesos a memoria de la implementación C (no es exactamente esa cantidad pues los ajustes de
fin de línea introducen un pequeño adicional de lecturas, dependiendo del ancho de la imagen.)
Notemos que este número se condice con las mediciones de rendimiento realizadas.

\subsection{Halftone}

La idea de utilizar assembler fue principalmente aprobechar mejor las operaciones existentes de SIMD, teniendo en cuenta que en un registro xmm de $128$ bits entran $16$ bytes y que cada pixel de una imagen en escala de grises ocupa exactamente un byte; podriamos pensar en trabajar con 16 pixeles de la imagen de manera simultanea. 

Para entender mejor lo implementado en assembler primero presentaremos un pseudocodigo del algoritmo:

//Pseudocodigo

El primer problema con el que nos encontramos mirando el pseudocodigo, es que teniendo en cuenta que cada pixel puede tener un número entre el $0$ y el $255$, al obtener la suma de $4$ pixeles podemos encontrarnos en ciertos casos con que dicha suma supere el valor $255$ y no lo podamos reprecentar con un solo byte. Es por esto que tuvimos que desempaquetar los $16$ bytes a word, alojando la parte baja en un registo y la parte alta en otro. De esta manera, seguimos teniendo $16$ pixeles de la imagen, pero ya no de forma simultanea, sino duplicando los pasos, ya que cada instrucción tiene que ser llamada dos veces, una para cada registro.

El segundo problema con el que nos encontramos es que la imagen es una matriz de pixeles de dimensiones desconocidas, y podriamos encontrarnos con casos donde el alto o el ancho de la misma contenga una cantidad impar de pixeles. Como el filtro funciona con bloques de $2x2$, y el enunciado dice que la imagen de destino tiene que tener dimensiones pares, lo que hacemos en estos casos, es decrementar en uno el alto y/o ancho si estos si son impares.

Por otro lado, en algunos pasos del algoritmo fue necesario la utilización de mascaras y valores predefinidos, las cuales eran siempre los mismos sin importar en que iteración estemos trabajando, es por esto que las calculamos al principio antes de empezar a recorrer la imagen, para de esta manera ahorrar llamados a memoria, los cuales reducen significativamente la velocidad de procesamiento de la aplicación.

estos valores son:
\begin{itemize}
  	\item $xmm9$ $\Rightarrow$ contiene en sus dos quadword el valor $0xffffffffffffffff$
	\item $xmm10$ $\Rightarrow$ contiene en sus dos quadword el valor $0x00ff00ff00ff00ff$
	\item $xmm11$ $\Rightarrow$ contiene en sus dos quadword el valor $0xff00ff00ff00ff00$
	\item $xmm12$ $\Rightarrow$ $205$,$205$,$205$,$205$,$205$,$205$,$205$,$205$
	\item $xmm13$ $\Rightarrow$ $410$,$410$,$410$,$410$,$410$,$410$,$410$,$410$
	\item $xmm14$ $\Rightarrow$ $615$,$615$,$615$,$615$,$615$,$615$,$615$,$615$
	\item $xmm15$ $\Rightarrow$ $820$,$820$,$820$,$820$,$820$,$820$,$820$,$820$ 
\end{itemize}

\subsubsection{Descripción del ciclo:}

\begin{itemize}
	\item El primer paso realizado dentro de un ciclo es obtener los proximos $16$ pixeles de la fila actual de la imagen en el registro $xmm0$, y los $16$ pixeles correspondientes de la fila siguiente en el registro $xmm2$, para luego trabajar con $8$ bloques de $2x2$ a la misma vez. 

	\begin{figure}[H]
		\centering
		\includegraphics[width=15cm]{halftone/bloques.png}
		\caption{obtención de los bloques a procesar en la iteración actual.}
		\label{bloques}
	\end{figure}

	\item El segundo paso consiste en desempaquetar los datos utilizando la instrucción PUNPCKHBW y PUNPCKLBW. Obteniendo de esta manera el valor de los pixeles en word's en los registros $xmm0$,$xmm1$ los de la fila actual y en $xmm2$,$xmm3$ los de la siguiente fila.
	Es importante destacar en este paso, que la parte baja del registro $xmm0$ es desempaquetada en el mismo registro, miestras que la parte alta se desempaqueta guardando el resultado en el registro $xmm1$. lo mismo pasa con $xmm2$,$xmm3$, teniendo en cuenta que el valor del pixel es un Integer Unsigned, al desempaquetar completamos con $0$'s y seguimos teniendo el mismo número. ver figura PONER EL LABEL

	\begin{figure}[H]
		\centering
		\includegraphics[width=15cm]{halftone/punpack.png}
		\caption{Desempaquetado de los bytes a words.}
		\label{PUNPCK}
	\end{figure}

	\item El tercer paso del ciclo es sumar el valor de cada bloque de $2x2$. Para esto utilizamos $2$ instrucciones, PADDW y PHADDW. La primer instrucción es utilizada para sumar los valores de los pixeles de la fila actual con los valores de los pixeles correspondientes en la siguiente fila, obteniendo de esta manera en los registros $xmm0$ y $xmm1$ las sumas parciales de los bloques. O de otra forma; siendo $(i,j)$ y $(i+1,j)$ la posiciones de la imagen desde donde se obtienen los proximos $32$ bytes a procesar, lo que obtenemos en el word más bajo de $xmm0$ es la suma del pixel $(i,j)$ con el pixel $(i+1,j)$ y asi hasta el word más alto de $xmm1$ donde tenemos la suma entre el pixel $(i,j+15)$ con el pixel $(i+1,j+15)$. ver figura PONER LABEL

	\begin{figure}[H]
		\centering
		\includegraphics[width=15cm]{halftone/PADDW.png}
		\caption{suma parcial de los bloques que se estan procesando.}
		\label{PADDW}
	\end{figure}


	La segunda instrucción (PHADDW) es utilizada para obtener la suma total de cada uno de los bloques. Tomando como ejemplo el bloque compuesto por los pixeles $(i,j)$,$(i,j+1)$,$(i+1,j)$ y $(i+1,j+1)$, ya tenemos las sumas parciales de los mismos en el registro $xmm0$. Más precisamente, tenemos en el word mas bajo del registro la suma de los pixeles $(i,j)$ y $(i+1,j)$ y el el siguiente word mas bajo la suma de los pixeles $(i,j+1)$ y $(i+1,j+1)$, por lo que al sumar estos dos word que se encuentran de manera consecutiva en el registro $xmm0$, lo que obtenemos es la suma total del bloque. 

	Dado que cada registro xmm tiene $8$ word y que cada bloque esta compuesto por dos word consecutivos del mismo registro, la instrucción PHADD utilizando como operandos los registros $xmm0$ y $xmm1$ devuelven en el registro $xmm0$ efectivamente la suma total para cada uno de los $8$ bloques que se estan procesando en simultaneo. ver figura PONER LABEL

	\begin{figure}[H]
		\centering
		\includegraphics[width=15cm]{halftone/PHADDW.png}
		\caption{suma total de los bloques que se estan procesando.}
		\label{PHADDW}
	\end{figure}

	\item El cuarto paso consiste en reemplazar cada bolque de $2x2$ por el bloque que le corresponda dependiendo de el valor obtenido en la suma total del mismo.

	Para esto utilizamos los registros $xmm12$ al $xmm15$, en los cuales estan guardados en paquet word los valores $205$,$410$,$615$ y $820$.

	Mirando como son los bloques predefinidos en el enunciado, y teniendo en cuenta que el valor $255$ que reprecenta al blanco en un byte es $0xff$ y el del negro es $0x0$, armamos dos mascaras con los valores $0x00ff00ff00ff00ff00ff00ff00ff00ff$ y $0xff00ff00ff00ff00ff00ff00ff00ff00$. Estas mascaras dejan los valores ya seteados en bytes, por lo que luego no hay que empaquetarlos.
	Para entender bien la idea miremos observemos primero lo siguiente:

	Cada pixel de la imagen ocupa un byte, pero como arrancamos de la posición $0$ y siempre nos movemos una cantidad par de posiciones en cada iteración (en particular nos movemos de a $16$ pixeles), sabemos que en los registros donde guardaremos el resultado para mandar al destino, cada par contiguo de pixeles pertenese a un mismo bloque de $2x2$, dado que la suma total de cada bloque también está guardada en word's, y que dos pixeles contiguos ocupan un word, podemos ver que la suma de un bloque abarca a todos los pixeles del mismo. ver figura PONER LABEL

	\begin{figure}[H]
		\centering
		\includegraphics[width=15cm]{halftone/masckBloques.png}
		\caption{relación entre el espacio ocupado por la suma total de un bloque y los pixeles del mismo en bytes.}
		\label{masckBloques}
	\end{figure}

	Luego, obtenemos la mascara resultante de comparar la suma total de los boques que se estan procesando con los valores $205$,$410$,$615$ y $820$. Esta mascara está en word's empaquetados, en los cuales hay unos o ceros dependiendo de si el word pertenece a un bloque cuya suma es mayor o menor que los valores con los que es comparado. 

	Sabiendo que para todo bloque mayor a $205$ el primer pixel de la primer fila debe ser blanco. a la mascara de comparaciones le aplicamos la mascara con los valores  $0xff00ff00ff00ff00ff00ff00ff00ff00$ y lo que obtuvimos fue que para aquellos boques mayores a $205$ los bytes de posiciones pares tienen unos mientras que los impares tienen ceros. Como vimos recien que el byte tenga unos es lo mismo a que tenga un $255$ por lo que de esta manera ya tenemos en la mascaraa resultante los valores deseados para las posiciones pares de la primer fila y ceros en las posiciones impares.

	Esto mismo es realizado para los valores $410$,$615$ y $801$ obteniendo en $4$ registros xmm los valores de cada pixel procesado, luego para unir estos valores, se hace un POR entre los dos registros pertenecientes a la primer fila y otro POR entre los dos registros pertenecientes a la segunda fila y asi obtenemos el resultado ya listo para mandar a memoria. Ver figura PONER LABEL

	\begin{figure}[H]
		\centering
		\includegraphics[width=15cm]{halftone/masckResult.png}
		\caption{creacion de las mascaras y obtención del resultado a guardar.}
		\label{masckResult}
	\end{figure}

	La instrucción utilizada para esta comparación es PCMPGTW.

	\item Este ultimo paso consiste en guardar en la memoria correspondiente los pixeles ya procesados.

\subsubsection{Comparación con el lenguaje C}
Mirano el pseudocodigo del algoritmo en C se puede ver que la principal diferencia es que mientras que en un ciclo del codigo en C se procesa un solo bloque, en assembler estamos procesando 8. Pero ademas de esto, para obtener los pixeles de un bloque en C tenemos que hacer 4 llamados a memoria mientras que en assembler para obtener los 8 bloques hacemos 2 llamados en memoria.	Esto mismo se repite a la hora de guardar los datos en la imagen de destino.

Por otro lado teniendo en cuenta ciertas propiedades, como el hecho de que un byte con todos unos es un byte con un $255$ en Unsigned Integer, en la version assembler sabemos que al crear la mascara ya tenemos colocado el valor deseado. El compilador no utiliza tales propiedades como tampoco utiliza instrucciones SIMD. Es por esto que ademas de utilizar instrucciones de comparación agrega instrucciónes para insertar los valores deseados, generando mas lentitud en el algoritmo. 

Si nos ponemos a ver con un poco mas de detalles, para realizar la suma del bloque que se esta procesando, en C se utilizan cuatro repeticiones de la instrucción ADD por cada bloque, por lo cual se realizan $32$ ADD para obtener la suma de 8 bloques, mientras que al utilizar SIMD, con 2 instrucciones PADDW y una instruccion PHADDW obtenemos las mismas sumas.



\subsection{Umbralizar}

El filtro umbralizar consiste en evaluar cada pixel de la imagen y asignarle un nuevo valor según tres criterio.

\begin{itemize}
\item Si el pixel supera el máximo pasado por parámetro a la función, se le coloca un 255.
\item Si el pixel es menor que el mínimo, también pasado por parámetro, se le coloca un 0.
\item Si el pixel está entre ($min \leq pixel \leq max$), se le asigna $\lfloor pixel/Q \rfloor * Q$, donde $Q$ es un parámetro.
\end{itemize}


\subsubsection{Descripción del ciclo}

El filtro se divide en cinco etapas:

\begin{enumerate}
\item \textbf{Pre-ciclo:} Se crean ciertos registros que serán de utilidad en el ciclo.
\item \textbf{Inicio del ciclo:} Puesta en 0 del registro acumulador y obtención de máscara de mínimos.
\item \textbf{Obtención de la máscara para píxeles mayores al máximo y aplicación:} Se consigue la máscara para los píxeles que superen al mayor y se aplica al acumulador.
\item \textbf{Creación de la máscara para los píxeles ($min \leq pixel \leq max$):} Se arma la máscara a partir de una nueva comparación y máscaras anteriores.
\item \textbf{Aplicación de la máscara para ($min \leq pixel \leq max$) y fin del ciclo:} Se realizan los cálculos pertinentes a los píxeles que entran en esta categoría y lo aplica a la máscara.
\end{enumerate}


\subsubsection{Pre-ciclo}

Para optimizar el procesamiento dentro del ciclo, se calcula y se guarda en registros XMM al inicio del programa:
\begin{itemize}
  \item \xmm{12} $\Rightarrow$ Contiene 16 bytes packed con el mínimo
  \item \xmm{11} $\Rightarrow$ Contiene el mínimo packed en words
  \item \xmm{5} $\Rightarrow$ Contiene el máximo packed en words
  \item \xmm{6} $\Rightarrow$ Contiene la representación flotante de Q packed
\end{itemize}

También se calcula y se guarda en el registro \rcx, la cantidad de píxeles que tiene la imagen, información que se utilizará para conocer cuántos píxeles faltan por procesar en cada iteración.

\subsubsection{Inicio del ciclo}
Al comienzo del ciclo se ponen en 0 los bytes del registro \xmm{8} que servirá como acumulador de los nuevos valores que tendrán los píxeles.

Se leen 16 bytes contiguos desde la imagen fuente en el registro \xmm{1} utilizando la instrucción \texttt{MOVDQU}, se busca cuáles son iguales al mínimo (aprovechando la instrucción \asm{PCMPEQB} que realiza la comparación en los 16 bytes simultáneamente) y se guarda la máscara obtenida que se usará mas tarde.

\subsubsection{Obtención de la máscara para píxeles mayores al máximo y aplicación}

Queriendo explotar al máximo las instrucciones SIMD, tratando de procesar en simultaneo 16 bytes, encontramos que esto no era posible debido a que el set de instrucciones no contempla comparaciones de greater, lower o derivadas para bytes sin signo (necesario ya que los píxeles en grayscale van del 0 al 255), por lo que fue necesario extender los bytes mediante un desempaquetado, convirtiéndolos en words, para luego hacer las comparaciones correspondientes.

Se desempaquetan los 16 bytes en parte baja (primeros 8 bytes) y en parte alta convirtiéndolos a words por medio de la instrucción \texttt{PUNPCKLBW}.

\begin{center}
  \begin{tikzpicture}[scale=0.75]
    \registroDieciseis{\xmm{1}}{0}{6}
                      {A15}{A14}{A13}{A12}{A11}{A10}{A9}{A8}{A7}
                      {A6}{A5}{A4}{A3}{A2}{A1}{A0}

    \draw [decoration={brace,amplitude=0.25cm},decorate, thick] ( 7.75, 5.75) -- (0.25, 5.75);
    \draw [decoration={brace,amplitude=0.25cm},decorate, thick] (15.75, 5.75) -- (8.25, 5.75);

    \draw [->, thick] (12, 5.5)  --  +(0, -1.25);
    \draw [->, thick] (4, 5.5)   -- ++(0, -0.75) -- ++(-6, 0) --
                    ++(0, -2.75) -- ++(6, 0)     --  +(0, -0.75);

    \registroDieciseis{\xmm{1}}{0}{3}
                      {0}{A7}{0}{A6}{0}{A5}{0}{A4}{0}{A3}{0}{A2}{0}{A1}{0}{A0}

    \registroDieciseis{\xmm{2}}{0}{0}
                      {0}{A15}{0}{A14}{0}{A13}{0}{A12}{0}{A11}{0}{A10}{0}{A9}{0}{A8}
  \end{tikzpicture}
\end{center}

Se buscan los números que superen al máximo comparando la parte alta y baja con el registro \xmm{5} preparado al inicio del programa, el cuál contiene el máximo en words empaquetadas. Utilizando \texttt{PCMPGTW} tanto en la parte baja como la alta, obtenemos una máscara que contiene en \texttt{0xFFFF} las words mayores al máximo y en \texttt{0x0000} las demás. Empaquetamos, utilizando la instrucción \texttt{PACKSSWB}, la máscara relacionada a la parte baja y a la alta, dejando en 255 sólo los bytes que sean mayores al máximo para luego sumarlos en el acumulador utilizando \texttt{PADDUSB}.

\begin{center}
  \begin{tikzpicture}[scale=0.75]
    \draw (-1, 15) node[anchor=west] {Estado inicial de los registros:};

    \registroDieciseis{\xmm{5}}{0}{13}
                      {0}{Max}{0}{Max}{0}{Max}{0}{Max}{0}{Max}{0}{Max}{0}{Max}{0}{Max}

    \registroDieciseis{\xmm{4}}{0}{11}
                      {0}{A15}{0}{A14}{0}{A13}{0}{A12}{0}{A11}{0}{A10}{0}{A9}{0}{A8}

    \registroDieciseis{\xmm{3}}{0}{9}
                      {0}{A7}{0}{A6}{0}{A5}{0}{A4}{0}{A3}{0}{A2}{0}{A1}{0}{A0}

    \draw (-1, 8) node[anchor=west]
      {Luego de ejecutar \texttt{PCMPGTW} \xmm{3}, \xmm{5} y \texttt{PCMPGTW} \xmm{4}, \xmm{5}:};

    \registroDieciseis{\xmm{4}}{-1}{6}
                      {0}{0}{0}{0}{0}{0}{0}{0}{0}{0}{0}{0}{FF}{FF}{FF}{FF}

    \registroDieciseis{\xmm{3}}{1}{4}
                      {0}{0}{0}{0}{FF}{FF}{FF}{FF}{0}{0}{FF}{FF}{0}{0}{FF}{FF}

    \draw [->, thick] (12, 3.75)   --  +(0, -1.75);

    \draw [->, thick] (-0.5, 5.75) -- ++(0, -3.25) -- ++ (4.5, 0) -- + (0, -0.5);

    \draw [decoration={brace,amplitude=0.25cm},decorate, thick] (0.25, 1.25) -- ( 7.75, 1.25);
    \draw [decoration={brace,amplitude=0.25cm},decorate, thick] (8.25, 1.25) -- (15.75, 1.25);

    \draw (-1, 3) node[anchor=west, fill=white]
      {Luego de ejecutar \texttt{PACKSSWB} \xmm{3}, \xmm{4}:};

    \registroDieciseis{\xmm{3}}{0}{0}
                      {0}{0}{0}{0}{0}{0}{FF}{FF}{0}{0}{FF}{FF}{0}{FF}{0}{FF}

  \end{tikzpicture}
\end{center}



\subsubsection{Creación de la máscara para los píxeles ($min \leq pixel \leq max$)}
Al utilizar un acumulador incialmente en 0 se pudo ahorrar el paso de comparar los píxeles menores al mínimo ya que los píxeles que cumplieran esta propiedad tendrían 0 como valor.

Para conseguir la máscara, se aprovechó las máscaras previamente obtenidas (mayores al máximo e iguales al mínimo), por lo que sólo fue necesario buscar los números mayores al mínimo (haciendo un procedimiento similar al de la búsqueda del máximo), agregarle los números iguales al mínimo (mediante un \texttt{POR}) con la máscara creada al principio del ciclo y luego sacarle los mayores al máximo (mediante un \texttt{PXOR}) dejándome sólo los píxeles que cumplen esta condición.

\subsubsection{Aplicación de la máscara para ($min \leq pixel \leq max$) y fin del ciclo}
Al ser necesaria un división y un truncamiento para el caso, se utilizaron single precision floats.

Fue necesario desempaquetar aún más los píxeles, utilizando \texttt{PUNPCKLWD}, para poder obtener los valores de estos en double words, y poder convertirlos a single precision floats mediante la instrucción \texttt{CVTDQ2PS}. Esta es precisión suficiente para los cálculos y además brinda la posibilidad realizar más operaciones sobre los píxeles simultáneamente que con doubles.

Se realiza el desempaquetado de la parte baja anteriormente desempaquetada (\xmm{1}) y se los convierte a single presicion floats. Luego se los divide, utilizando \texttt{DIVPS}, por el registro \xmm{6} el cuál contiene el valor Q empaquetado en single presicion floats calculado al inicio del programa. Se lo trunca utilizando \texttt{CVTTPS2DQ}, para realizar la función floor ($\lfloor \rfloor$) convirtiéndose en entero, se lo convierte otra vez a float y se lo termina multiplicando, utilizando \texttt{MULPS}, por \xmm{6} otra vez para finalmente ser convertido a entero, obteniendo el valor correspondiente para cada pixel. Se vuelve a empaquetar, utilizando la instrucción \texttt{PACKUSDW}, obteniéndose otra vez los valores nuevos de los píxeles en words y se repite la operación con la parte alta del desempaquetado inicial (\xmm{2}).

Al finalizar, se empaquetan los dos registros resultantes, mediante \texttt{PACKUSWB}, para luego poder aplicarle la máscara anteriormente creada para el caso utilizando un \texttt{PAND} y sumar simultáneamente con \texttt{PADDUSB} el valor obtenido al acumulador, dejando el valor correspondiente en los píxeles que cumplían con el caso.

Por último, se copian los nuevos valores de los 16 bytes en la imagen destino utilizando \texttt{MOVDQU} y se incrementan los punteros de la imagen fuente y destino para la próxima iteración. También se resta nuestro contador de píxeles por procesar, se lo compara para ver si llegó al final, caso en el que termina la ejecución, o si quedan más e igual, caso en el que vuelve a ciclar, o si restan menos de 16 bytes por agarrar, dónde se vuelve para atrás los punteros de las imágenes para que queden 16 píxeles exactamente y se pueda realizar la última iteración.

\subsubsection{Comparación con la implementación C}
El ciclo en C hace esencialmente lo mismo que la implementación en assembler, pero esta lo hace de manera simultanea utilizando SIMD. 

Estos son los detalles de las operaciones realizadas en C para 16 bytes:
\begin{itemize}
\item 16 accesos a memoria para la lectura de cada byte y 16 para la escritura
\item Se realizan a lo sumo 32 comparaciones, (2 por cada pixel), en dónde se necesita calcular la posición del pixel cada vez que se lo quiere leer y se realiza un acceso a memoria para esto.
\item En caso de haber caído dentro del caso ($min \leq pixel \leq max$), se realizan cada una de las operaciones de cálculo de manera secuencial.
\end{itemize}
Detalles de las operaciones realizadas por la implementación de assembler en 16 bytes:
\begin{itemize}
\item Única lectura y escritura en memoria de 16 bytes
\item Se realizan 2 comparaciones para el caso de los píxeles mayores al máximo y para calcular los píxeles mayores al mínimo por cada 16 bytes, ya que se comparan de a 8 bytes simultáneamente utilizando SIMD. También se realiza 1 comparación cada estos 16 bytes para obtener los iguales al mínimo.
\item Si bien siempre se realiza el cálculo en punto flotante relacionado a los valores de los píxeles del caso ($min \leq pixel \leq max$), estos se realizan con 4 instrucciones SIMD, pudiendo calcular el valor de 16 bytes por ciclo.
\end{itemize}

\subsubsection{Rendimiento}
Observamos las siguientes cantidades de ciclos y ticks de reloj al realizar 100 iteraciones de ambas implementaciones con una imagen cuadrada de lado 512 y parámetros $min = 64$, $max = 128$ y $Q = 16$.
\begin{center}
    \begin{tabular}{|l|l|l|l|}
        \hline
        Medición & Implementación C & Implementación assembler & Relación \\
        \hline
        Ticks    & 785867600      & 70172072               & $8.92\%$ \\
        Ciclos   & 7858676        & 701721                & $8.92\%$ \\
        \hline
    \end{tabular}
\end{center}

Para hacer un análisis fino, utilizamos la herramienta objdump (\texttt{objdump -d -M intel -S umbralizar\_c.o}) para obtener el resultado de la compilación mediante gcc.

Destacamos de este análisis el uso contínuo de variables locales (almacenadas en el stack) y el cálculo de la posición en memoria y el acceso a esta cada vez que se quería realizar una comparación, algo evitable usando registros en ensamblador y que seguro impactaron en el rendimiento.

\begin{verbatim}
if(src_matrix[y][x] < min) {
;Instrucciones para settear la posición del byte
mov    eax,DWORD PTR [rbp-0x4c]
movsxd rdx,eax
movsxd rax,ebx
imul   rdx,rax
mov    rax,QWORD PTR [rbp-0x38]
add    rdx,rax
mov    eax,DWORD PTR [rbp-0x48]
cdqe   
movzx  eax,BYTE PTR [rdx+rax*1]
;Termina el setteado de la posición del byte
cmp    al,BYTE PTR [rbp-0x70] ;Realiza la comparación
jae    c3 <umbralizar_c+0xc3>
\end{verbatim}

Cabe destacar que el lenguaje ensamblador, además del uso de las instrucciones SIMD, que nos permiten realizar cálculos simultáneamente otorgándonos una gran ventaja contra el código en C, también nos permite explotar los recursos al máximo, pudiendo obviar cálculos de más guardándolos para su posterior uso y utilizar registros que implican menos cantidad de accesos a memoria.

Estas ventajas, sumadas a los pocos accesos de memoria al poder escribir y leer de a 16 bytes son la causa del bajo tiempo de ejecución del filtro en assembler.

\subsection{Colorizar}

El filtro colorizar modifica cada pixel de una imagen color de acuerdo a una ecuación que depende
de los máximos valores de los canales R, G y B entre los píxeles del cuadrado de lado 3 formado
por el pixel original y sus 8 vecinos inmediatos.

El problema principal a resolver es minimizar la cantidad de accesos a memoria y paralelizar lo más
posible las comparaciones necesarias para hallar el máximo valor de cada canal en el cuadrado
de lado 3 correspondiente a cada pixel.


%%%%%%%%%%%%%%%%%%%%%%%%%%%%%%%%%%%%%%%%%%%%%%%%%%%%%%%%%%%%%%%%%%%%%%%%%%%%%%%
%% Descripción del ciclo                                                     %%
%%%%%%%%%%%%%%%%%%%%%%%%%%%%%%%%%%%%%%%%%%%%%%%%%%%%%%%%%%%%%%%%%%%%%%%%%%%%%%%


\subsubsection{Descripción del ciclo}

Para cada pixel, dividimos el ciclo en cuatro etapas:

\begin{enumerate}
    \item \textbf{Lectura de píxeles vecinos:}
        Cargamos los píxeles de cada fila del cuadrado en \xmm{1}, \xmm{2} y \xmm{3}.
    \item \textbf{Búsqueda de máximos por canal:}
        Separamos cada pixel por canal, y comparamos sucesivamente los valores de cada canal
        entre sí para hallar los máximos.
    \item \textbf{Hallar $\Phi_{R,G,B}$:}
        Hallamos los valores de $\Phi_R$, $\Phi_G$ y $\Phi_B$ utilizando los máximos
        obtenidos en la etapa anterior.
    \item \textbf{Escritura de pixel destino:}
        Computamos el valor del pixel destino y lo escribimos en la imagen.
\end{enumerate}

Veamos a continuación cada etapa en detalle.


%%%%%%%%%%%%%%%%%%%%%%%%%%%%%%%%%%%%%%%%%%%%%%%%%%%%%%%%%%%%%%%%%%%%%%%%%%%%%%%
%% Lectura de píxeles                                                        %%
%%%%%%%%%%%%%%%%%%%%%%%%%%%%%%%%%%%%%%%%%%%%%%%%%%%%%%%%%%%%%%%%%%%%%%%%%%%%%%%


\subsubsection{Lectura de píxeles vecinos}

Cargamos los píxeles de cada fila del cuadrado de lado 3 como se indica a continuación:

\begin{pseudocodigo}
    \STATE \texttt{MOVDQU} \xmm{1}, $src[y-1][x-3]$
    \STATE \texttt{MOVDQU} \xmm{2}, $src[y][x-3]$
    \STATE \texttt{MOVDQU} \xmm{3}, $src[y+1][x-3]$        
\end{pseudocodigo}

Contenido de los registros \xmm{1}, \xmm{2} y \xmm{3} en este instante:

\begin{figure}[H]
    \centering
    \begin{tikzpicture}[scale=0.75]
        \registroDieciseis{}{0}{0}
                          {$\ast$}{$\ast$}{$\ast$}{$\ast$}{$\ast$}{$\ast$}{$\ast$}
                          {$R$}{$G$}{$B$}{$R$}{$G$}{$B$}{$R$}{$G$}{$B$}
    \end{tikzpicture}
    \caption{Contenido de cada registro luego de leer los píxeles vecinos.}
\end{figure}

El símbolo $\ast$ representa información de píxeles adyacentes que no nos interesa.

En el caso de estar leyendo los últimos píxeles de la última fila, las lecturas
se realizan desplazándonos hacia atrás sobre el eje X tantos píxeles como sea
necesario, de forma tal que la parte alta de cada registro XMM esté alineada
con el último pixel. En este caso el contenido de cada registro sería
el siguiente:

\begin{figure}[H]
    \centering
    \begin{tikzpicture}[scale=0.75]
        \registroDieciseis{}{0}{0}
                          {$R$}{$G$}{$B$}{$R$}{$G$}{$B$}{$R$}{$G$}{$B$}
                          {$\ast$}{$\ast$}{$\ast$}{$\ast$}{$\ast$}{$\ast$}{$\ast$}
    \end{tikzpicture}
    \caption{Ajuste de fin de última fila.}
\end{figure}

Luego de realizar esta lectura, reacomodamos los bytes con \texttt{PSHUFB}
para tenerlos en el mismo orden que en el caso normal, y procedemos normalmente.

%%%%%%%%%%%%%%%%%%%%%%%%%%%%%%%%%%%%%%%%%%%%%%%%%%%%%%%%%%%%%%%%%%%%%%%%%%%%%%%
%% Búsqueda de máximos                                                       %%
%%%%%%%%%%%%%%%%%%%%%%%%%%%%%%%%%%%%%%%%%%%%%%%%%%%%%%%%%%%%%%%%%%%%%%%%%%%%%%%


\subsubsection{Búsqueda de máximos por canal}

El siguiente paso es comparar los valores de cada canal entre sí para hallar los máximos.

Lo primero que hacemos es reordenar los bytes en \xmm{1}, \xmm{2} y \xmm{3}:

\begin{pseudocodigo}
    \STATE \texttt{PSHUFB} \xmm{1}, \textit{máscara}
    \STATE \texttt{PSHUFB} \xmm{2}, \textit{máscara}
    \STATE \texttt{PSHUFB} \xmm{3}, \textit{máscara}
\end{pseudocodigo}

La \textit{máscara} provista es tal que los bytes de los registros quedan
reordeados de la siguiente forma:

\begin{figure}[H]
    \centering
    \begin{tikzpicture}[scale=0.75]
        \registroDieciseis{}{0}{0}
                          {0}{0}{0}{0}
                          {0}{$R$}{$R$}{$R$}
                          {0}{$G$}{$G$}{$G$}
                          {0}{$B$}{$B$}{$B$}
    \end{tikzpicture}
    \caption{Reordenamiento de canales.}
\end{figure}

A continuación, buscamos los máximos byte a byte entre \xmm{1} y \xmm{2} y luego entre
\xmm{1} y \xmm{3} de manera de obtener los máximos por canal ``columna a columna'':

\begin{pseudocodigo}
    \STATE \texttt{PMAXUB} \xmm{1}, \xmm{2}
    \STATE \texttt{PMAXUB} \xmm{1}, \xmm{3}    
\end{pseudocodigo}

La siguiente ilustración muestra lo que ocurre con los primeros 3 bytes de
\xmm{1}, \xmm{2} y \xmm{3}, que contienen los valores del canal B para los
tres píxeles de las filas 1, 2 y 3 respectivamente:

\begin{figure}[H]
    \centering
    \begin{tikzpicture}[scale=0.75]
        \foreach \i in {1, ..., 3}
        {
            \draw (0,  {-(\i - 1) * 2 + 0.5}) node[anchor=east]{\xmm{\i}};
            \draw (0,  {-(\i - 1) * 2}) rectangle +(1, 1) +(0.5, 0.5) node{$\dots$};
            \draw (1,  {-(\i - 1) * 2}) rectangle +(5, 1) +(2.5, 0.5) node{$B_{{\i}c}$};
            \draw (6,  {-(\i - 1) * 2}) rectangle +(5, 1) +(2.5, 0.5) node{$B_{{\i}b}$};        
            \draw (11, {-(\i - 1) * 2}) rectangle +(5, 1) +(2.5, 0.5) node{$B_{{\i}a}$};
        }

        \draw (0, -5) node[anchor=west]
            {Luego de ejecutar \texttt{PMAXUB} \xmm{1}, \xmm{2} y
                               \texttt{PMAXUB} \xmm{1}, \xmm{3}:};

        \draw (0, {-7 + 0.5}) node[anchor=east]{\xmm{1}};
        \draw (0,  -7) rectangle +(1, 1) +(0.5, 0.5) node{$\dots$};
        \draw (1,  -7) rectangle +(5, 1) +(2.5, 0.5) node{$max(B_{1c}, B_{2c}, B_{3c})$};
        \draw (6,  -7) rectangle +(5, 1) +(2.5, 0.5) node{$max(B_{1b}, B_{2b}, B_{3b})$};        
        \draw (11, -7) rectangle +(5, 1) +(2.5, 0.5) node{$max(B_{1a}, B_{2a}, B_{3a})$};        
    \end{tikzpicture}
    \caption{Búsqueda de máximos columna a columna.}
\end{figure}

Lo mismo ocurre con el resto de los bytes que contienen los valores para los canales R y G.

En este instante el contenido de \xmm{1} es el siguiente:

\begin{figure}[H]
    \centering
    \begin{tikzpicture}[scale=0.75]
        \registroDieciseis{\xmm{1}}{0}{0}
                          {0}{0}{0}{0}
                          {0}{$R_c$}{$R_b$}{$R_a$}
                          {0}{$G_c$}{$G_b$}{$G_a$}
                          {0}{$B_c$}{$B_b$}{$B_a$}
    \end{tikzpicture}
    \caption{Resultados de la búsqueda de máximos columna a columna.}
\end{figure}

Donde $B_a = max(B_{1a}, B_{2a}, B_{3a})$, $\dots$, $R_c = max(R_{1c}, R_{2c}, R_{3c})$.

Para encontrar $max_R$, $max_G$ y $max_B$ aún nos falta maximizar las tuplas
$(R_a, R_b, R_c)$, $(G_a, G_b, G_c)$ y $(B_a, B_b, B_c)$.

Extraemos las tres triplas $(R, G, B)$ del registro \xmm{1} usando
\texttt{PSHUFB} con las máscaras correspondientes y las almacenamos por separado en
\xmm{1}, \xmm{2} y \xmm{3}, de manera de tener un único valor R, G y B en cada uno de
estos registros, para poder buscar los máximos en paralelo columna a columna con
\texttt{PMAXUB}, como hicimos antes.

Gráficamente, podemos representar esta serie de pasos de la siguiente manera:

\begin{figure}[H]
    \centering
    \begin{tikzpicture}[scale=0.75, >=stealth]
        \registroDieciseis{\xmm{1}}{0}{10}
                          {0}{0}{0}{0}
                          {0}{$R_c$}{$R_b$}{$R_a$}
                          {0}{$G_c$}{$G_b$}{$G_a$}
                          {0}{$B_c$}{$B_b$}{$B_a$}

        \draw [->, thick] (15.5, 10) -- (15.5, 8); % Ba
        \draw [->, thick] (14.5, 10) -- (15.5, 6); % Bb
        \draw [->, thick] (13.5, 10) -- (15.5, 4); % Bc

        \draw [->, thick] (11.5, 10) -- (11.5, 8); % Ga
        \draw [->, thick] (10.5, 10) -- (11.5, 6); % Gb
        \draw [->, thick]  (9.5, 10) -- (11.5, 4); % Gc

        \draw [->, thick]  (7.5, 10) --  (7.5, 8); % Ra
        \draw [->, thick]  (6.5, 10) --  (7.5, 6); % Rb
        \draw [->, thick]  (5.5, 10) --  (7.5, 4); % Rc        

        \draw (0, 9) node[anchor=west, fill=white]{Luego de extraer las triplas:};

        \registroDieciseis{\xmm{1}}{0}{7}
                          {0}{0}{0}{0}
                          {0}{0}{0}{$R_a$}
                          {0}{0}{0}{$G_a$}
                          {0}{0}{0}{$B_a$}

        \registroDieciseis{\xmm{2}}{0}{5}
                          {0}{0}{0}{0}
                          {0}{0}{0}{$R_b$}
                          {0}{0}{0}{$G_b$}
                          {0}{0}{0}{$B_b$}

        \registroDieciseis{\xmm{3}}{0}{3}
                          {0}{0}{0}{0}
                          {0}{0}{0}{$R_c$}
                          {0}{0}{0}{$G_c$}
                          {0}{0}{0}{$B_c$}

        \draw (0, 2) node[anchor=west]
            {Luego de ejecutar \texttt{PMAXUB} \xmm{1}, \xmm{2} y
                               \texttt{PMAXUB} \xmm{1}, \xmm{3}:};

        \registroDieciseis{\xmm{1}}{0}{0}
                          {0}{0}{0}{0}
                          {0}{0}{0}{$R^\ast$}
                          {0}{0}{0}{$G^\ast$}
                          {0}{0}{0}{$B^\ast$}
    \end{tikzpicture}
    \caption{Extracción de triplas $(R, G, B)$.}
\end{figure}

Donde $R^\ast = max_R$, $G^\ast = max_G$ y $B^\ast = max_B$.

Finalmente extraemos los máximos con \texttt{PSHUFD} y \texttt{MOVD},
y los guardamos en registros de propósito general, para usarlos en la
siguiente etapa.


%%%%%%%%%%%%%%%%%%%%%%%%%%%%%%%%%%%%%%%%%%%%%%%%%%%%%%%%%%%%%%%%%%%%%%%%%%%%%%%
%% Hallar Phi                                                               %%
%%%%%%%%%%%%%%%%%%%%%%%%%%%%%%%%%%%%%%%%%%%%%%%%%%%%%%%%%%%%%%%%%%%%%%%%%%%%%%%


\subsubsection{Hallar $\Phi_{R,G,B}$}

Procedemos a computar $\Phi_R$, $\Phi_G$ y $\Phi_B$ de forma secuencial,
luego empaquetamos en \xmm{1} los valores hallados como se indica a continuación:

\begin{figure}[H]
    \centering
    \begin{tikzpicture}[scale=0.75]
        \registroCuatro{\xmm{1}}{0}{0}{$\ast$}{$\Phi_R$}{$\Phi_G$}{$\Phi_B$}
    \end{tikzpicture}
    \caption{Empaquetado de los valores $\Phi_{R,G,B}$ computados.}
\end{figure}

Empaquetamos $\Phi_{R,G,B}$ de esta manera para facilitar las
comparaciones que realizaremos en la siguiente y última etapa.


%%%%%%%%%%%%%%%%%%%%%%%%%%%%%%%%%%%%%%%%%%%%%%%%%%%%%%%%%%%%%%%%%%%%%%%%%%%%%%%
%% Pixel destino                                                             %%
%%%%%%%%%%%%%%%%%%%%%%%%%%%%%%%%%%%%%%%%%%%%%%%%%%%%%%%%%%%%%%%%%%%%%%%%%%%%%%%


\subsubsection{Escritura de pixel destino}

Antes de escribir el pixel destino, debemos hallar el valor de cada canal.
Recordemos las ecuaciones:

\begin{center}
    $I_{dst_R}(i, j) = min(255, \Phi_R \ast I_{src_R}(i, j))$ \\
    $I_{dst_G}(i, j) = min(255, \Phi_G \ast I_{src_G}(i, j))$ \\
    $I_{dst_B}(i, j) = min(255, \Phi_B \ast I_{src_B}(i, j))$
\end{center}

Procedemos a hallar $I_{src_{R,G,B}}(i, j)$:

\begin{pseudocodigo}
    \STATE \texttt{MOVDQU} \xmm{2},   $src[j][i]$      \COMMENT{leo el pixel original}
    \STATE \texttt{PSHUFB} \xmm{2},   \textit{máscara} \COMMENT{reacomodo y limpio registro}
    \STATE \texttt{CVTDQ2PS} \xmm{2}, xmm{2}           \COMMENT{convierto cada canal a float}
\end{pseudocodigo}

En este instante \xmm{2} contiene el valor de cada canal del pixel original
convertido a punto flotante como se ilustra a continuación:

\begin{figure}[H]
    \centering
    \begin{tikzpicture}[scale=0.75]
        \registroCuatro{\xmm{2}}{0}{0}{$\ast$}{$I_{src_R}$}{$I_{src_G}$}{$I_{src_B}$}
    \end{tikzpicture}
    \caption{Canales R, G, B convertidos a punto flotante.}
\end{figure}

Multiplicamos cada canal por el valor $\Phi$ correspondiente y buscamos
el mínimo entre este producto y 255:

\begin{pseudocodigo}
    \STATE \texttt{MULPS}    \xmm{2}, \xmm{1} \COMMENT{\xmm{2} $\leftarrow$ $\Phi \ast I_{src}$}
    \STATE \texttt{MOVDQU}   \xmm{1}, \textit{tupla 255}
        \COMMENT{cargo el valor 255 en los 3 floats más bajos de \xmm{1}}
    \STATE \texttt{MINPS}    \xmm{1}, \xmm{2}
        \COMMENT{\xmm{1} $\leftarrow$ $min(255, \Phi \ast I_{src}) = I_{dst}$}
    \STATE \texttt{CVTPS2DQ} \xmm{1}, \xmm{1} \COMMENT{convertimos a enteros los $I_{dst}$ hallados}
\end{pseudocodigo}

El contenido de de \xmm{1} es ahora:

\begin{figure}[H]
    \centering
    \begin{tikzpicture}[scale=0.75]
        \registroCuatro{\xmm{1}}{0}{0}{$\ast$}{$I_{dst_R}$}{$I_{dst_G}$}{$I_{dst_B}$}
    \end{tikzpicture}
    \caption{Valores del pixel destino por canal.}
\end{figure}

Si bien los valores $I_{dst_{R,G,B}}$ resultantes de la conversión de punto flotante a
entero sin signo son de 32 bits, recordemos que nunca se exceden de 255, por lo que podemos
asumir que nuestro dato está en el byte más bajo de cada entero, mientras que en los bytes
restantes hay ceros. Entonces el contenido de \xmm{1} es:

\begin{figure}[H]
    \centering
    \begin{tikzpicture}[scale=0.75]
        \registroDieciseis{\xmm{1}}{0}{0}
                          {$\ast$}{$\ast$}{$\ast$}{$\ast$}
                          {0}{0}{0}{$I_{dst_R}$}
                          {0}{0}{0}{$I_{dst_G}$}
                          {0}{0}{0}{$I_{dst_B}$}
    \end{tikzpicture}
    \caption{Valores del pixel destino por canal representados cada uno en un único byte.}
\end{figure}

Reordenamos los bytes con \texttt{PSHUFB}:

\begin{figure}[H]
    \centering
    \begin{tikzpicture}[scale=0.75]
        \registroDieciseis{\xmm{1}}{0}{0}
                          {$\ast$}{$\ast$}{$\ast$}{$\ast$}
                          {0}{0}{0}{0}{0}{0}{0}{0}
                          {0}{$I_{dst_R}$}{$I_{dst_G}$}{$I_{dst_B}$}
    \end{tikzpicture}
    \caption{Construcción de la tira de bytes del pixel destino.}
\end{figure}

Para terminar, extraemos $I_{dst_R}$, $I_{dst_G}$ y $I_{dst_B}$ con \texttt{MOVD}
y escribimos el pixel obtenido en la imagen destino.


%%%%%%%%%%%%%%%%%%%%%%%%%%%%%%%%%%%%%%%%%%%%%%%%%%%%%%%%%%%%%%%%%%%%%%%%%%%%%%%
%% Comparación con la implementación C                                       %%
%%%%%%%%%%%%%%%%%%%%%%%%%%%%%%%%%%%%%%%%%%%%%%%%%%%%%%%%%%%%%%%%%%%%%%%%%%%%%%%


\subsubsection{Comparación con la implementación C}

Escencialmente, la versión C de este filtro realiza la misma secuencia de operaciones,
pero accediendo a memoria de a un byte por vez, y realizando las comparaciones
sin paralelismo. Esto significa que se realizan:

\begin{itemize}
    \item $9 * 3 = 27$ accesos a memoria para obtener cada canal de los píxeles vecinos,
    \item 3 accesos a memoria para leer el pixel original,
    \item 3 accesos a memoria para escribir el pixel destino, y
    \item 24 comparaciones para hallar $max_{R,G,B}$.
\end{itemize}

Esto es un total de 33 accesos a memoria y 24 comparaciones.

En contraste, la implementación assembler realiza:

\begin{itemize}
    \item 3 accesos a memoria para obtener los píxeles vecinos,
    \item 1 acceso a memoria para leer el pixel original,
    \item 1 acceso a memoria para escribir el pixel destino, y
    \item 6 comparaciones para hallar $max_{R,G,B}$.
\end{itemize}

Es decir, un total de 5 accesos a memoria y 6 comparaciones.

Expresado de forma relativa, la implementación assembler realiza
el $15.2\%$ de accesos a memoria y el $25\%$ de comparaciones que su
contraparte en lenguaje C.


%%%%%%%%%%%%%%%%%%%%%%%%%%%%%%%%%%%%%%%%%%%%%%%%%%%%%%%%%%%%%%%%%%%%%%%%%%%%%%%
%% Rendimiento                                                               %%
%%%%%%%%%%%%%%%%%%%%%%%%%%%%%%%%%%%%%%%%%%%%%%%%%%%%%%%%%%%%%%%%%%%%%%%%%%%%%%%


\subsubsection{Rendimiento}

Observamos las siguientes cantidades de ciclos y ticks de reloj al realizar 100 iteraciones de ambas implementaciones con una imagen cuadrada de lado 512 y parámetro $\alpha = 0.5$:

\begin{center}
    \begin{tabular}{|l|l|l|l|}
        \hline
        Medición & Implementación C & Implementación assembler & Relación \\
        \hline
        Ticks    & 53062000890      & 1213052436               & $2.28\%$ \\
        Ciclos   & 530620000        & 12130524                 & $2.28\%$ \\
        \hline
    \end{tabular}
\end{center}

En la sección anterior evaluamos las relaciones de accesos a memoria y comparaciones
entre ambas implementaciones. Sin realizar un mayor análisis, éstas nos sirven
para tener una idea intuitiva de las mejoras que podríamos esperar de haber optimizado
utilizando instrucciones SIMD. Sin embargo el rendimiento obtenido fue casi un orden
de magnitud superior.

Luego de analizar el código assembler generado por el compilador, identificamos una
gran cantidad de accesos a memoria adicionales, producidos por el uso de variables locales
almacenadas en la pila. Esto ocurre tanto en el ciclo principal como en varios auxiliares.
Atribuímos la diferencia de performance a estos accesos a memoria que no tuvimos en cuenta
en nuestro análisis inicial.

\subsection{Efecto plasma}

Este filtro consiste en obtener los pixeles de la imagen original y aplicarle una serie de cuentas para luego insertar el resultado en la imagen de destino. Las cuentas que se realizan son las siguientes. \\

\begin{itemize}
	\item $prof = ( x\_scale*sin\_taylor(i/8.0) + y\_scale*sin\_taylor(j/8.0) )/2$. Donde $sin\_taylor$ es una función que calcula el seno de $x$ a travez de una aproximación, y $x\_scale$,$y\_scale$ son parámetros de entrada de la función.
	\item Si el valor optenido es mayor a $255$ el resultado es $255$.
	\item Si el contenido es menor a $0$ el resultado es $0$.
	\item En otro caso el se trunca el resultado.
\end{itemize}

\subsubsection{Pseudo codigo del algoritmo en C:}

\begin{pseudocodigo}
   \FOR {alto de la imagen}
		\FOR {ancho de la imagen}
			\STATE
			\STATE $prof = ( x\_scale*sin\_taylor(i/8.0) + y\_scale*sin\_taylor(j/8.0) )/2$ 
			\STATE $newValue = prof*g\_scale + src\_matrix[i][j]$ 
			\STATE
			\IF {$newValue > 255$}
				\STATE $newValue = 255$
			\ELSE 
				\IF{$newValue < 0$}
					\STATE $newValue = 0$
				\ENDIF 
			\ENDIF
			\STATE $dst\_matrix[i][j] = floor(newValue)$ 
		\ENDFOR
	\ENDFOR
\end{pseudocodigo}

$sin\_taylor(x)$:


\begin{pseudocodigo}
   \FOR {alto de la imagen}
		\FOR {ancho de la imagen}
			\STATE
			\STATE $pi	= 3.14159265359$
			\STATE $k 			= floor(x/(2*pi))$
			\STATE $r 		= x - k*2*pi$
			\STATE $x 				= r - pi$
			\STATE $y 		= x - (pow(x,3)/6) + (pow(x,5)/120) - (pow(x,7)/5040)$
			\STATE
			\STATE devolver $y$
			\STATE
		\ENDFOR
	\ENDFOR
\end{pseudocodigo}

\subsubsection{Idea general:}
La motivación de utilizar assembler para la resolución de esta aplicación es fundamentalmente la aplicación de las herramientas SIMD para mejorar la performanse del algoritmo. Teniendo en cuenta que el mismo trabaja haciendo iteraciónes sobre los píxeles de una imagen y procesando los mismos, si podemos procesar más de un pixel por iteración mejoramos de forma significativa la performance de la aplicación.\\

La idea principal fue implementar el algoritmo utilizando los registros $xmm$ para poder procesar 16 píxeles por iteración. Ante esta idea, el problema que nos surgio fue el siguiente:\\

Como se puede ver en el pseudocodigo debemos llamar a la función $sin\_taylor(j/80.0)$ y $sin\_taylor(i/80.0)$, donde $(i,j)$ hacen referencia a la posición del pixel dentro de la matriz de la imagen. No podemos asumir nada en cuanto al valor de $j$ e $i$, incluso, estos pueden ser tan grandes como quieran, dependiendo el tamaño de la imagen, haciendo en ciertos casos que el algoritmo en assembler tenga fallas si se superara la precisión de un byte. 

El algoritmo en C esta implementado con float, lo cual es una precisión razonable para las dimensiones de imagenes con que se suele trabajar, es por esto que decidimos trabajar los datos con doubleword's, codificandolos como punto flotante de simple precisión.

Al trabajar con los datos en esta precisión lo que tenemos es que para trabajar con $16$ pixeles necesitamos $4$ registros $xmm$ para cargar los indices de las columnas a los que estos pertenecen, más otro registro para cargar el valor empaquetado del indice de la fila a la que pertencen. Si miramos la función $sin\_taylor$ veremos que necesitamos entonces, 5 registros para el acumulador de la cuenta, 5 registros para mantener el valor original mientras que en otros 5 registros se van calculando cada termino del polinomio. Esto nos da una suma de $15$ registros $xmm$. hay maneras de poder trabajar con 16 pixeles reutilizando registros y llamando a memoria, pero como justamente no queremos hacer llamados a memoria, nos parecio más eficiente procesar 8 pixeles por iteración y utilizar los registros $xmm$ sobrantes para precargar valores que seran utilizados en todas las iteraciones. \\

\subsubsection{Pre-ciclo principal:}
una vez visto esto, pasemos a nombrar todos los registros que pre-seteamos antes de empezar el ciclo, y que luego serán utilizado dentro del mismo.

estos valores son:
\begin{itemize}
	\item $xmm0$ $\Rightarrow$ $x\_scale$;$x\_scale$;$x\_scale$;$x\_scale$
	\item $xmm1$ $\Rightarrow$ $y\_scale$;$y\_scale$;$y\_scale$;$y\_scale$
	\item $xmm2$ $\Rightarrow$ $g\_scale$;$g\_scale$;$g\_scale$;$g\_scale$
	\item $xmm3$ $\Rightarrow$ $pi$;$pi$;$pi$;$pi$
	\item $xmm4$ $\Rightarrow$ $2.0$;$2.0$;$2.0$;$2.0$
	\item $xmm5$ $\Rightarrow$ $3$;$2$;$1$;$0$
	\item $xmm6$ $\Rightarrow$ $7$;$6$;$5$;$4$
	\item $xmm7$ $\Rightarrow$ $6.0$,$6.0$;$6.0$;$6.0$
	\item $xmm8$ $\Rightarrow$ $120.0$;$120.0$;$120.0$;$120.0$
	\item $xmm9$ $\Rightarrow$ $5040.0$;$5040.0$;$5040.0$;$5040.0$
\end{itemize}

\subsubsection{Descripción del ciclo:}
\begin{itemize}
	\item El primer paso realizado dentro de la iteración es guardar en $2$ registros $xmm$ los valores de los indices de los pixeles que se van a procesar, Sea $(i,j)$ el primero de los $8$ pixeles que seran procesados, los valores que se necesitan guardar son: $j$,$j+1$,$j+2$,$j+3$,$j+4$,$j+5$,$j+6$,$j+7$,$j+8$, y también el valor de $i$, el cual es el mismo para todos los pixeles ya que estamos trabajando sobre una misma fila.
	
	Para hacer esto, necesitamos los siguientes pasos:
	\begin{itemize}
	 	\item Copiamos el valor de los registros $xmm5$ y $xmm6$ a los registros $xmm10$ y $xmm11$ utilizando la instrucción MOVDQU, obteniendo en ellos los valores del $0$ al $7$.
	 	\item En un registro $xmm$ se le pone en todos sus doubleword's el valor de $j$ utilizando la instrucciones MOVQ que mueve un doubleword en la parte baja del registro $xmm$ y extiende el resto del registro con $0$'s y SHUFPS que lo expande en las demas doubleword's del registro. 
	 	\item Se lo codifica a punto flotante con la instrucción CVTDQ2PS. 
	 	\item Luego se procede a sumarle a los registros $xmm10$ y $xmm11$ el registro donde empaquetamos el valor de $j$, para esto Utilizamos la instrucción ADDPS que realiza la suma empaquetada de doubleword codificados como puntos flotantes.
	\end{itemize}
	
	(El algoritmo procesa mas adelante el valor de $i$, por lo que en esta instancia no hacemos nada con el).

	\item El segundo paso a realizar es dividir los valores de los $j$'s por $8$, pero como tenemos ya en un registro $xmm$ empaquetado el número $2.0$ como doubleword's, lo que hacemos es dividir tres veces por este registro utilizando la intrucion DIVPS.

	\subsubsection{función $Sin\_taylor$:}
	\item  El próximo paso es implementarle a estos valores la funcion $sin\_taylor$. Para esto copiamos a los registros $xmm12$ y $xmm13$ el contenido de los registros $xmm10$,$xmm11$ respectivamente, y seguimos los siguientes pasos:

	\begin{itemize}
		\item Dividimos los valores de cada pixel por $2*pi$. Como tenemos estos valores en los registros $xmm3$ y $xmm4$ utilizamos nuevamente las intrucciones DIVPS para dividir $xmm12$ y $xmm3$ por los registros anteriores, y luego aplicamos las instrucciones CVTTPS2DQ y CVTTPS2DQ para obtener la parte entera.

		\item El proximo paso de la funcion de taylor es restarle a los valores originales el resultado anterior previamente multiplicado por $2*pi$. Primero, multiplicamos utilizando la instrucción MULPS, los registros $xmm3$ y $xmm4$ a los registros $xmm12$,$xmm13$ donde estan los resultados anteriores, y luego a los registros $xmm10$ y $xmm11$ donde se encuentran los valores originales le restamos los registros $xmm12$,$xmm13$ correspondientementes utilizando la instrucción  SUBPS.\\
		En este paso hemos perdido los valores originales que teniamos empaquetados en $xmm10$ y $xmm11$ pero esto no nos importa ya que no los necesitamos más.

		\item Luego vamos a restarle a lo recien obtenido el valor de $pi$ empaquetado en el registro $xmm3$, y a continuación una seguidilla de pasos repetidos de forma casi igual, por lo que consideramos mejor una explicación general para estos pasos, y no una lectura tediosa y repetitiva de la realización de los mismos.

		Estos pasos son utilizados para obtener el valor siguiente, $Y = X - X^3/6 + X^5/120 - X^7/5040$ en cada doubleword de dos registros $xmm$ de forma empaquetada ($X$ hace referencia a los valores obtenidos mediante los pasos anteriores que estan empaquetados en los registros $xmm12$ y $xmm13$).
		Para realizar esta cuenta, se guarda el valor de $xmm12$,$xmm13$ en dos registros $xmm$ para poder salvarlos, luego se copian de vuelta estos valores en otros dos registros $xmm$ que seran utilizados como acumulador. y lo siguiente a realizar es ir multiplicando y dividiendo estos registros para obtener cada uno de los terminos del polinomio y sumarselo o restarselo a los registros acumuladores para ir teniendo el valor del polinomio en los acumuladores. Los dividendos utilizados estan ya pre-seteados en los registros $xmm7$,$xmm8$,$xmm9$ de forma empaquetada codificados en punto flotante.
	\end{itemize}

	\item Una vez que tenemos el $sin\_taylor$ para los indices $j$'s de forma empaquetada en los registros $xmm12$,$xmm13$, pasamos a empaquetar el indice $i$ en el registro $xmm10$ y realizamos los mismos pasos anteriores para obtener el $sin\_taylor$ de $i/8.0$. Con lo cual al final de los pasos conseguimos en el registro $xmm11$ el valor empaquetado en doubleword's de $sin\_taylor$ para el indice $i/8.0$, (el cual es el mismo valor para todos los pixeles por pertenecer a la misma fila).

	\item Lo siguiente, es obtener el valor de $prof(i,j)$. Para esto, primero le multiplicamos a los registro $xmm12$,$xmm13$ donde estan empaquetados los taylor's de los indices $j$'s$/8.0$ el registro $xmm1$, que es donde esta empaquetado el valor de $y\_scale$, y le multiplicamos  a $xmm11$ el registro $xmm0$ ya que en este esta empaquetado el valor de $x\_scale$. Una vez realisado esto, le sumamos a los registros $xmm12$,$xmm13$ el registro $xmm11$ y los dividimos por el registro $xmm4$ que es el que tiene el valor $2.0$ empaquetado en doubleword.

	De esta menera queda guardado en los registros $xmm12$,$xmm13$ la funcion $Prof(i,j)$ de los indices de los pixeles que se estan procesando. En $xmm12$ estan los valores para los pixeles de las pisiciones $(i,j)$ hasta $(i,j+3)$ y en $xmm13$ estan los valores para los pixeles de las posiciones $(i,j+4)$ hasta $(i,j+7)$

	\item El siguiente paso consiste en traer los valores de los pixeles de la memoria y desempaquetarlos guardando el valor en los registros $xmm14$ y $xmm15$ (en $xmm14$ se guardan los primeros $4$ pixeles y en $xmm15$ el resto).

	Teniendo estos valores hacemos la suma de $xmm12$ con $xmm14$ y $xmm13$ con $xmm15$ y tenemos el valor ya listo para ser empaquetado y guardado en el destino.

	Teniendo en cuenta que este valor puede ser mayor a $255$ o menor a $0$ para empaquetar utilizamos la instrucción PACKUSDW que empaqueta saturando integer's sin signo, dejando asi en $255$ todos aquellos valores mayores al mismo y en $0$ a los valores negativos.

\end{itemize}

\subsubsection{Comparación con el lenguaje C}

Mirando el pseudocodigo del algoritmo en C vemos para empezar que la cantidad de iteraciones es 8 veces mayor en C que en assembler al utilizar SIMD. ya que mientras que en C se procesa de a un solo pixel por vez en assembler se procede a procesar 8.

Otro aspecto a tener en cuenta es que en la implementación en assembler tenemos que obtener a partir de la posición del primer pixel los valores de los $j$'s para los otros $7$ pixeles que se van a procesar, si vien esto en C no hay que hacerlo, el costo de este paso del algoritmo esta esta compensado ya que al realizar 8 veces menos cantidad de iteraciones hay que avanzar ocho veces menos los contadores. Podemos concluir entonces que en ciertos casos, si vien paralelizar supone realizar algunos pasos extras, hay que ver si los mismos son compensados o no. pueden existir casos donde la cantidad de pasos extras a compensar sean demasiados y termine bajando la performance comparandolo con algoritmos comunes.

Tambien podemos ver que en la version C cada vez que se llama a la funcion $sin\_taylor$, se define el valor de $pi$ mientras que en la implementacion assembler el mismo ya esta predefinido antes de comenzar el ciclo, lo mismo pasa con los valores $6$, $120$ y $5040$. esto mejora la performance, pero no se lo puede considerar como una mejora de SIMD ya que lo mismo podria ser facilmente aplicado en la version C del algoritmo.\\

\underline{\textbf{Consideraciones a la hora de comparar la versión C con assembler:}}

Tengamos en cuenta para la comparación de ambas implementaciones que en C por cada pixel se hace la llamada a la funcion $sin\_taylor$ que como sabemos supone ciertos pasos para satisfacer la convencion C y pasarle los parametros a la misma, Esto lo convierten mas lento que si la misma funcion estuviera definida dentro del mismo algoritmo.

También pasa que tanto en la implementación C como en la implementación assembler, podriamos mejorarlas calculando la funcion $sin\_taylor$($i/8.0$) una sola vez por cada fila, de esta manera se reduce significativamente la llamada a la función, mejorando la performanze. En este punto es importante ver que esta mejora es $8$ veces menos significatica para assembler que para C, esto se debe a que en el mismo ya se procesan 8 pixeles por vez. Por lo que vemos que mismo en algoritmos ineficientes puede pasar que al paralelizar estemos obteniendo de por si un algoritmo mejorado y aplicable a la versión no paralelizada del mismo sin siquiera saberlo.

\subsubsection{Rendimiento}

Observamos las siguientes cantidades de ciclos y ticks de reloj al realizar 100 iteraciones de ambas implementaciones con una imagen cuadrada de lado 512.
\begin{center}
    \begin{tabular}{|l|l|l|l|}
        \hline
        Medición & Implementación C & Implementación assembler & Relación \\
        \hline
        Ticks    & 40908368256      & 960464334            & $2.35\%$ \\
        Ciclos   & 409083712        & 9604643              & $2.35\%$ \\
        \hline
    \end{tabular}
\end{center}

\subsection{Rotar}

El filtro rotar es el menos capacitado para paralelización. Esta dificultad proviene de la imposibilidad de traer bytes y procesarlos simultaneamente, debido a que el valor de cada pixel en la imagen destino se obtiene calculando, según su posición, qué pixel corresponde de la imagen fuente, lo que implica traer uno a uno los pixeles de la imagen fuente para colocarlos en en lugar indicado en la imágen destino.

Analizando el algoritmo, llegamos a la conclusión de que la mejor manera de explotar la simultaneidad que nos brinda las instrucciones SIMD, es utilizarlas para calcular las posiciones de dónde sacar el pixel para colorcarlo en la imagen destino. 

Inicialmente este proceso fue realizado utilizando single precision floats para poder aprovechar SIMD al máximo y realizar cálculos para conseguir las posiciones de donde sacar los pixeles a la vez. Este intento se vió frustrado al correr el filtro y compararlo contra las imágenes provistas por la cátedra. Si bien la imagen coincidía visualmente, se encontraron errores en 27 pixeles. Un análisis de estos pixeles pusieron en evidencia un problema de precisión, el cual se intentó arreglar utilizando la instrucción roundps sin conseguir mejores resultados. Se solucióno el problema realizando los cálculos usando double precision floats con SIMD para luego convertirlos a floats y finalmente convertirlos a enteros para poder funcionar como posición.

Para optimizar la implementación, se realiza el cálculo de cx, cy y $\sqrt[2]{2}/2$ al inicio del programa.

El programa recorre las posiciones de la imagen destino y les asigna su valor según su posición.

\subsubsection{Descripción del ciclo}
El ciclo comienza poniendo en 0 mediante la instrucción por el registro xmm7 el cuál se encargará de acumular 16 bytes con los valores correspondientes de los píxeles.

 El algoritmo continúa con un pequeño ciclo interno el cuál se encarga de poner en 2 registros temporales los valores de las posiciones x e y de 4 pixeles sucesivos. Una vez realizado esto, para poder trabajar con los 4 pixeles utilizando doubles, se mueve el contenido de estos registros a registros xmm dónde se los convierte y se los almacena 4 registros xmm, 2 con los 2 doubles más bajos y altos de las coordenadas x y 2 con los 2 doubles más bajos y altos de las coordenadas y.

Se opera sobre los doubles utilizando instrucciones SIMD, obteniedo los valores u y v, es decir, los valores x e y respectivamente de dónde conseguir el pixel a poner en la imágen fuente.

Al finalizar el cálculo se convierten de doubles a floats y de floats a enteros y vuelven a colocar las coordenadas x en un registro xmm y las coordenadas y en otro cómo estaban antes.

Después de esta conversión, se pasa a un pequeño ciclo el cuál se encarga de tomar cada coordenada, ver si cumplen las restricciones del cálculo para u y v, y colocar en el caso de que se cumpla, en un acumulador, el valor del pixel que tenga en la posición en la imagen fuente y si no, un 0.

Al terminar este ciclo, se agregan los valores de estos 4 bytes al el acumulador principal que se creó al comienzo del ciclo y se vuelve a ciclar realizando el mismo proceso hasta llenar el acumulador con los valores correspondientes de los 16 bytes o 16 pixeles contiguos.

En el final de cada iteración del ciclo princiapl, se checkea la posición de la imagen en la que se está (por si ya se ha terminado de recorrerla o se es necesario correr para atrás para agarrar 16 pixeles exactos) y si se ha llenado el acumulador. Si el acumulador se llena (habiendo pasado 4 ciclos) se mueven los 16 bytes a la imagen destino, economizando el movimiento de datos.

\subsubsection{Comparación con el lenguaje C}



%%%%%%%%%%%%%%%%%%%%%%%%%%%%%%%%%%%%%%%%%%%%%%%%%%%%%%%%%%%%%%%%%%%%%%%%%%%%%%%
%% Resultados                                                                %%
%%%%%%%%%%%%%%%%%%%%%%%%%%%%%%%%%%%%%%%%%%%%%%%%%%%%%%%%%%%%%%%%%%%%%%%%%%%%%%%


\section{Resultados}

Resumimos a continuación las mejoras de rendimiento obtenidas al aplicar instrucciones SIMD. El gráfico que sigue presenta el porcentaje del tiempo de ejecución de las implementaciones assembler respecto de las implementaciones C. Cuanto más pequeña sea la barra, mayor es la mejora de rendimiento obtenida.

\begin{figure}[H]
    \centering
    \begin{tikzpicture}[>=stealth]
        % Porcentajes
        \def\recortar  {6.43}
        \def\halftone  {6.64}
        \def\umbralizar{8.92}
        \def\colorizar {2.28}
        \def\plasma    {2.35}
        \def\rotar     {41.17}

        \def\escala    {0.2}

        % Recortar
        \filldraw[fill=gray] (1,    0) rectangle ++(1, {\recortar * \escala});
        \draw                (1.5,  0) node[anchor=north]{Recortar};
        \draw                (1.5, {\recortar * \escala}) node[anchor=south]{\recortar\%};

        % Halftone
        \filldraw[fill=gray] (3,    0) rectangle +(1, {\halftone * \escala});
        \draw                (3.5,  0) node[anchor=north]{Halftone};
        \draw                (3.5, {\halftone * \escala}) node[anchor=south]{\halftone\%};

        % Umbralizar
        \filldraw[fill=gray] (5,    0) rectangle +(1, {\umbralizar * \escala});
        \draw                (5.5,  0) node[anchor=north]{Umbralizar};
        \draw                (5.5, {\umbralizar * \escala}) node[anchor=south]{\umbralizar\%};

        % Colorizar
        \filldraw[fill=gray] (7,    0) rectangle +(1, {\colorizar * \escala});
        \draw                (7.5,  0) node[anchor=north]{Colorizar};
        \draw                (7.5, {\colorizar * \escala}) node[anchor=south]{\colorizar\%};

        % Plasma
        \filldraw[fill=gray] (9,    0) rectangle +(1, {\plasma * \escala});
        \draw                (9.5,  0) node[anchor=north]{Plasma};
        \draw                (9.5, {\plasma * \escala}) node[anchor=south]{\plasma\%};

        % Rotar
        \filldraw[fill=gray] (11,   0) rectangle +(1, {\rotar * \escala});
        \draw                (11.5, 0) node[anchor=north]{Rotar};
        \draw                (11.5, {\rotar * \escala}) node[anchor=south]{\rotar\%};   

        % Ejes
        \draw [->] (0, 0) -- +(0,  10);
        \draw      (0, 0) -- +(13, 0);
    \end{tikzpicture}
    \caption{Tiempo de ejecución de las implementaciones assembler respecto a C.}
\end{figure}


%%%%%%%%%%%%%%%%%%%%%%%%%%%%%%%%%%%%%%%%%%%%%%%%%%%%%%%%%%%%%%%%%%%%%%%%%%%%%%%
%% Conclusión                                                                %%
%%%%%%%%%%%%%%%%%%%%%%%%%%%%%%%%%%%%%%%%%%%%%%%%%%%%%%%%%%%%%%%%%%%%%%%%%%%%%%%


\section{Conclusión}

A lo largo de este trabajo obtuvimos mejoras de rendimiento significativas de manera consistente al hacer uso de instrucciones SIMD. 

Identificamos dos posibles focos de aplicación para este juego de instrucciones: la paralelización de cómputos y el acceso a memoria de a bloques de datos contiguos. Observamos que en los casos que se logró acceder a más de un dato en memoria a la vez obtuvimos las ganancias de rendimiento más grandes.

En aquellos casos en los que no fue posible acceder a más de un dato en memoria a la vez, aún así obtuvimos grandes mejoras de rendimiento al paralelizar el procesamiento de los datos leídos en forma secuencial, aunque más pequeñas en comparación con los casos en los que esto sí fue posible.

Además, observamos que el código assembler generado por el compilador gcc (con nivel de optimización por defecto) produce una cantidad adicional de accesos a memoria al guardar en la pila los parámetros de las funciones y variables locales, que producen la penalidad de rendimiento correspondiente. Esto es especialmente evidente cuando el código C hace muchas llamadas a funciones, como en el caso de los filtros que invocan funciones auxiliares en el ciclo que itera sobre todos los píxeles de la imagen.


\end{document}