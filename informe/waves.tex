Este filtro combina la imagen original con una imagen de ondas, dando tonos más oscuros y más claros en forma de onda. Estas se generan de manera repetida a lo largo de los ejes x e y.

Veamos a continuación el ciclo de la implementación C:

\begin{pseudocodigo}
  \FOR{$i = 1$ to $alto$}
    \FOR{$j = 1$ to $ancho$}
      \STATE $prof$ $\leftarrow$ $(x_{scale} * sin\_taylor(i/8.0) + y_{scale} * sin\_taylor(j/8.0) )/2$ 
      \STATE $newValue$ $\leftarrow$ $prof * g_{scale} + src[i][j]$ 
      \IF{$newValue > 255$}
        \STATE $newValue$ $\leftarrow$ $255$
      \ELSE 
        \IF{$newValue < 0$}
          \STATE $newValue$ $\leftarrow$ $0$
        \ENDIF 
      \ENDIF
      \STATE $dst[i][j] = \lfloor newValue \rfloor$ 
    \ENDFOR
  \ENDFOR
\end{pseudocodigo}

Recordemos el algoritmo de $sin\_taylor$:

\begin{pseudocodigo}
  \STATE $k \leftarrow \lfloor x / 2\pi \rfloor$
  \STATE $r \leftarrow x - 2k\pi$
  \STATE $x \leftarrow r - \pi$
  \STATE $y \leftarrow x - \frac{x^3}{6} + \frac{x^5}{120} - \frac{x^7}{5040}$
  \STATE devolver $y$
\end{pseudocodigo}


\subsubsection{Idea general}

La idea principal fue implementar el algoritmo utilizando los registros XMM para poder procesar 16 píxeles por iteración. Ante esta idea, el problema que nos surgio fue el siguiente:

Como se puede ver en el pseudocódigo, debemos llamar a la función $sin\_taylor(j/80.0)$ y $sin\_taylor(i/80.0)$, donde $(i,j)$ hacen referencia a la posición del pixel dentro de la matriz de la imagen. No podemos asumir nada en cuanto al valor de $i$ y $j$, incluso, estos pueden ser tan grandes como quieran, dependiendo el tamaño de la imagen, haciendo en ciertos casos que el algoritmo en assembler tenga fallas si se superara la precisión de un byte. 

El algoritmo en C esta implementado con float, lo cual es una precisión razonable para las dimensiones de imágenes con que se suele trabajar, es por esto que decidimos trabajar los datos con doublewords, codificándolos como punto flotante de precisión simple.

Al trabajar con los datos en esta precisión lo que tenemos es que para trabajar con 16 píxeles necesitamos 4 registros XMM para cargar los índices de las columnas a los que estos pertenecen, más otro registro para cargar el valor empaquetado del índice de la fila a la que pertencen. Si miramos la función $sin\_taylor$ veremos que necesitamos entonces, 5 registros para el acumulador de la cuenta, 5 registros para mantener el valor original mientras que en otros 5 registros se van calculando cada termino del polinomio. Esto nos da una suma de 15 registros XMM. Hay maneras de poder trabajar con 16 píxeles reutilizando registros y llamando a memoria, pero como justamente no queremos hacer llamados a memoria, nos parecio más eficiente procesar 8 píxeles por iteración y utilizar los registros XMM sobrantes para precargar valores que seran utilizados en todas las iteraciones.


\subsubsection{Pre-ciclo principal}

Antes de entrar al ciclo, cargamos algunos valores en los registros XMM que nos serán útiles mientras se ejecuta el cuerpo del mismo:

\begin{itemize}
  \item \xmm{0} $\leftarrow$ $x_{scale}$ $x_{scale}$ $x_{scale}$ $x_{scale}$
  \item \xmm{1} $\leftarrow$ $y_{scale}$ $y_{scale}$ $y_{scale}$ $y_{scale}$
  \item \xmm{2} $\leftarrow$ $g_{scale}$ $g_{scale}$ $g_{scale}$ $g_{scale}$
  \item \xmm{3} $\leftarrow$ $\pi$ $\pi$ $\pi$ $\pi$
  \item \xmm{4} $\leftarrow$ $2.0$ $2.0$ $2.0$ $2.0$
  \item \xmm{5} $\leftarrow$ 3 2 1 0
  \item \xmm{6} $\leftarrow$ 7 6 5 4
  \item \xmm{7} $\leftarrow$ $6.0$ $6.0$ $6.0$ $6.0$
  \item \xmm{8} $\leftarrow$ $120.0$ $120.0$ $120.0$ $120.0$
  \item \xmm{9} $\leftarrow$ $5040.0$ $5040.0$ $5040.0$ $5040.0$
\end{itemize}


\subsubsection{Descripción del ciclo:}

\begin{itemize}
  \item El primer paso realizado dentro de la iteración es guardar en dos registros XMM los valores de los índices de los píxeles que se van a procesar. Sea $(i,j)$ el primero de los $8$ píxeles que seran procesados. Los valores que se necesitan guardar son: $j$, $j+1$, $j+2$, $j+3$, $j+4$, $j+5$, $j+6$, $j+7$, $j+8$, y también el valor de $i$, el cual es el mismo para todos los píxeles ya que estamos trabajando sobre una misma fila.
  
  Para hacer esto, necesitamos los siguientes pasos:
  \begin{itemize}
    \item Copiamos el valor de los registros \xmm{5} y \xmm{6} a los registros \xmm{10} y \xmm{11} utilizando la instrucción \asm{MOVDQU}, obteniendo en ellos los valores del $0$ al $7$.
    \item En un registro XMM se le pone en todos sus doublewords el valor de $j$ utilizando las instrucciones \asm{MOVQ} que mueve un doubleword en la parte baja del registro XMM y extiende el resto del registro con $0$ y SHUFPS que lo expande en las demás doublewords del registro. 
    \item Se lo codifica a punto flotante con la instrucción \asm{CVTDQ2PS}. 
    \item Luego se procede a sumarle a los registros \xmm{10} y \xmm{11} el registro donde empaquetamos el valor de $j$, para esto Utilizamos la instrucción \asm{ADDPS} que realiza la suma empaquetada de doubleword codificados como puntos flotantes.
  \end{itemize}
  
  (El algoritmo procesa mas adelante el valor de $i$, por lo que en esta instancia no hacemos nada con el).

  \item El segundo paso a realizar es dividir los valores de los $j$ por $8$, pero como tenemos ya en un registro XMM empaquetado el número $2.0$ como doubleword, lo que hacemos es dividir tres veces por este registro utilizando la instrucción \asm{DIVPS}.

  \item  El próximo paso es implementarle a estos valores la función $sin\_taylor$. Para esto copiamos a los registros \xmm{12} y \xmm{13} el contenido de los registros \xmm{10},\xmm{11} respectivamente, y seguimos los siguientes pasos:

  \begin{itemize}
    \item Dividimos los valores de cada pixel por $2\pi$. Como tenemos estos valores en los registros \xmm{3} y \xmm{4} utilizamos nuevamente las instrucciones \asm{DIVPS} para dividir \xmm{12} y \xmm{3} por los registros anteriores, y luego aplicamos las instrucciones \asm{CVTTPS2DQ} y \asm{CVTTPS2DQ} para obtener la parte entera.

    \item El próximo paso de la función de Taylor es restarle a los valores originales el resultado anterior previamente multiplicado por $2\pi$. Primero, multiplicamos utilizando la instrucción \asm{MULPS}, los registros \xmm{3} y \xmm{4} a los registros \xmm{12},\xmm{13} donde están los resultados anteriores, y luego a los registros \xmm{10} y \xmm{11} donde se encuentran los valores originales le restamos los registros \xmm{12}, \xmm{13} correspondientes utilizando la instrucción \asm{SUBPS}.

    En este paso hemos perdido los valores originales que teníamos empaquetados en \xmm{10} y \xmm{11} pero esto no nos importa ya que no los necesitamos más.

    \item Luego vamos a restarle a lo recién obtenido el valor de $pi$ empaquetado en el registro \xmm{3}, y a continuación una seguidilla de pasos repetidos de forma casi igual, por lo que consideramos mejor una explicación general para estos pasos, y no una lectura tediosa y repetitiva de la realización de los mismos.

    Estos pasos son utilizados para obtener el valor siguiente, $y = x - x^3/6 + x^5/120 - x^7/5040$ en cada doubleword de dos registros XMM de forma empaquetada ($x$ hace referencia a los valores obtenidos mediante los pasos anteriores que están empaquetados en los registros \xmm{12} y \xmm{13}).
    Para realizar esta cuenta, se guarda el valor de \xmm{12},\xmm{13} en dos registros XMM para poder salvarlos, luego se copian de vuelta estos valores en otros dos registros XMM que serán utilizados como acumulador, y lo siguiente a realizar es ir multiplicando y dividiendo estos registros para obtener cada uno de los términos del polinomio y sumárselo o restárselo a los registros acumuladores para ir teniendo el valor del polinomio en los acumuladores. Los dividendos utilizados están ya pre-seteados en los registros \xmm{7},\xmm{8},\xmm{9} de forma empaquetada codificados en punto flotante.
  \end{itemize}

  \item Una vez que tenemos el $sin\_taylor$ para los índices $j$ de forma empaquetada en los registros \xmm{12}, \xmm{13}, pasamos a empaquetar el índice $i$ en el registro \xmm{10} y realizamos los mismos pasos anteriores para obtener el $sin\_taylor$ de $i/8.0$. Con lo cual al final de los pasos conseguimos en el registro \xmm{11} el valor empaquetado en doubleword de $sin\_taylor$ para el índice $i/8.0$, (el cual es el mismo valor para todos los píxeles por pertenecer a la misma fila).

  \item Lo siguiente, es obtener el valor de $prof(i,j)$. Para esto, primero le multiplicamos a los registro \xmm{12}, \xmm{13} donde están empaquetados los Taylor de los índices $j/8.0$ el registro \xmm{1}, que es donde esta empaquetado el valor de $y_{scale}$, y le multiplicamos  a \xmm{11} el registro \xmm{0} ya que en este esta empaquetado el valor de $x_{scale}$. Una vez realizado esto, le sumamos a los registros \xmm{12}, \xmm{13} el registro \xmm{11} y los dividimos por el registro \xmm{4} que es el que tiene el valor $2.0$ empaquetado en doubleword.

  De esta manera queda guardado en los registros \xmm{12}, \xmm{13} la función $prof(i,j)$ de los índices de los píxeles que se están procesando. En \xmm{12} están los valores para los píxeles de las posiciones $(i,j)$ hasta $(i,j+3)$ y en \xmm{13} están los valores para los píxeles de las posiciones $(i,j+4)$ hasta $(i,j+7)$

  \item El siguiente paso consiste en traer los valores de los píxeles de la memoria y desempaquetarlos guardando el valor en los registros \xmm{14} y \xmm{15} (en \xmm{14} se guardan los primeros $4$ píxeles y en \xmm{15} el resto.)

  Teniendo estos valores hacemos la suma de \xmm{12} con \xmm{14} y \xmm{13} con \xmm{15} y tenemos el valor ya listo para ser empaquetado y guardado en el destino.

  Teniendo en cuenta que este valor puede ser mayor a $255$ o menor a $0$ para empaquetar utilizamos la instrucción \asm{PACKUSDW} que empaqueta saturando enteros sin signo, dejando así en $255$ todos aquellos valores mayores al mismo y en $0$ a los valores negativos.
\end{itemize}

\subsubsection{Comparación con la implementación C}

Mirando el pseudocódigo del algoritmo en C vemos para empezar que la cantidad de iteraciones es 8 veces mayor en C que en assembler al utilizar SIMD. ya que mientras que en C se procesa de a un solo pixel por vez en assembler se procede a procesar 8.

Otro aspecto a tener en cuenta es que en la implementación en assembler tenemos que obtener a partir de la posición del primer pixel los valores de los $j$ para los otros $7$ píxeles que se van a procesar, si bien esto en C no hay que hacerlo, el costo de este paso del algoritmo esta esta compensado ya que al realizar 8 veces menos cantidad de iteraciones hay que avanzar ocho veces menos los contadores. Podemos concluir entonces que en ciertos casos, si bien paralelizar supone realizar algunos pasos extras, hay que ver si los mismos son compensados o no. pueden existir casos donde la cantidad de pasos extras a compensar sean demasiados y termine bajando el rendimiento comparándolo con algoritmos comunes.

También podemos ver que en la versión C cada vez que se llama a la función $sin\_taylor$, se define el valor de $pi$ mientras que en la implementación assembler el mismo ya esta predefinido antes de comenzar el ciclo, lo mismo pasa con los valores $6$, $120$ y $5040$. esto mejora el rendimiento, pero no se lo puede considerar como una mejora de SIMD ya que lo mismo podría ser fácilmente aplicado en la versión C del algoritmo.

Veamos ahora algunas consideraciones a la hora de comparar la versión C con assembler. Tengamos en cuenta para la comparación de ambas implementaciones que en C por cada pixel se hace la llamada a la función $sin\_taylor$ que como sabemos supone ciertos pasos para satisfacer la convención C y pasarle los parámetros a la misma, Esto lo convierten mas lento que si la misma función estuviera definida dentro del mismo algoritmo.

También pasa que tanto en la implementación C como en la implementación assembler, podríamos mejorarlas calculando la función $sin\_taylor$($i/8.0$) una sola vez por cada fila, de esta manera se reduce significativamente la llamada a la función, mejorando el rendimiento. En este punto es importante ver que esta mejora es $8$ veces menos significativa para assembler que para C, esto se debe a que en el mismo ya se procesan 8 píxeles por vez. Por lo que vemos que mismo en algoritmos ineficientes puede pasar que al paralelizar estemos obteniendo de por si un algoritmo mejorado y aplicable a la versión no paralelizada del mismo sin siquiera saberlo.


\subsubsection{Rendimiento}

Observamos las siguientes cantidades de ciclos y ticks de reloj al realizar 100 iteraciones de ambas implementaciones con una imagen cuadrada de lado 512.
\begin{center}
    \begin{tabular}{|l|l|l|l|}
        \hline
        Medición & Implementación C & Implementación assembler & Relación \\
        \hline
        Ticks    & 40908368256      & 960464334            & $2.35\%$ \\
        Ciclos   & 409083712        & 9604643              & $2.35\%$ \\
        \hline
    \end{tabular}
\end{center}